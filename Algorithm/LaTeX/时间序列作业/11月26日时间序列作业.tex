\documentclass[12pt, a4paper]{ctexart}
\usepackage{fancyhdr}
\usepackage{graphicx}
\usepackage{harpoon}
\usepackage{ctex}
\usepackage{amsmath}
\pagestyle{plain}
\usepackage{mathrsfs}
\usepackage{amssymb}

\title{时间序列分析 作业}
\author{10161511403, 魏森辉}
\date{\today}
  

\begin{document}
\maketitle{}

\begin{flushleft}
    \qquad 题目一:
    设$\begin{bmatrix} x  \\ y \end{bmatrix}$为二维正态向量
    ,求用$y$对$x^2$的最小方差估计及线性无偏最小方差估计


    \qquad 答:(1) 设$[x, y]^T \sim  N(\mu_1,\theta_1 ^2, \mu_2, \theta_2 ^2, \rho)$ ,则$y$对$x$的最小方差估计是$E[x^2|y]$,令$z = x ^ 2$,我们有

    \[
    \begin{aligned}
    E[z|y] &= E[x^2|y] \\
    &= E^2(x|y) + Var(x|y) \\ 
    &=(\mu_1 + \rho \theta_1(y - \mu_2) / \theta_2 ^ 2 )^2 + \theta_1 ^2(1 - \rho^2)
    \end{aligned}
    \]
    
   (2)线性无偏最小方差估计$\hat{z}_{y} = Ez + R_{zy}R_{y}^{+}(y-Ey)$


    \[
    \begin{aligned}
     Ez &= VarX + (Ex)^2 = \theta_1 ^2 + \mu_1 ^2 \\
        R_y ^{+} &= \frac{1}{\theta_2 ^2}\\
    \end{aligned}
    \]


    \qquad 题目二:
    考虑如下的线性随机递推
    \[
    \left\{\begin{array}{ll}
    { x(k+1) = Ax(k) + D(\omega(k+1)) }  \\  % 加&增加列
    { y(k) = Cx(k)+F\omega(k) }
    \end{array}\right.
    \]
    \[
    x(k) \in \mathbb{R}^n, y(k) \in \mathbb{R}^n,\omega(k) \in \mathbb{R}^n 
    \]
    $\omega(k)$满足$\omega(k + 1) = M\omega(k) + \xi(k), M \in \mathbb{R}^{m \times m},\xi(k)$为零均值白噪声用$y^k$求$x(k),x(k + 1)$线性无偏最小方差估计,并求$Kalman$滤波方程

    \qquad 答:$y^k$对$x(k),x(k + 1)$的线性无偏最小方差估计为$\hat{x}(k | k), \hat{x} (k + 1 | k)$
    $Kalman$滤波方程不会求


\end{flushleft}

\end{document}

