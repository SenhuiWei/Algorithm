\documentclass[12pt, a4paper]{ctexart}
\usepackage{fancyhdr}
\usepackage{graphicx}
\usepackage{harpoon}
\usepackage{ctex}
\usepackage{amsmath}
\pagestyle{plain}
\usepackage{mathrsfs}
\usepackage{amssymb}
\usepackage{amsthm}
\usepackage[a4paper,left=3cm,right=3cm,top=2cm,bottom=2cm]{geometry} % paperwidth=11cm,scale=0.8
\usepackage{setspace} 
\renewcommand{\baselinestretch}{1.5} % 设置行间距的大小


\newtheorem{define}{定义} %在导言区使用,定义环境名

\title{\LaTeX 模版}
\date{\today}
\author{Wilson79}
  

\begin{document}
    \maketitle{} % 生成标题
    \tableofcontents % 产生文档目录
    \thispagestyle{empty} %目录页不显示页码
    \newpage
    \setcounter{page}{1} % 从下面开始编页码

   

\section{计算题}

    \begin{flushleft} % 设置左对齐
    {\bfseries 多项式求极限模型}
    
    看次数最高的项
    \[
    \lim _{n \rightarrow \infty} \frac{a_{m} n^{m}+a_{m-1} n^{m-1}+\cdots+a_{1} n+a_{0}}{b_{k} n^{k}+b_{k-1} n^{k-1}+\cdots+b_{1} n+b_{0}}=\left\{\begin{array}{ll}{\frac{a_{m}}{b_{m}},} & {k=m} \\ {0,} & {k>m}\end{array}\right.
    \] % \{是特殊字符,\left是定界符
    {\bfseries 和差化积-积化和差公式}

    \begin{align}
    \sin a + \sin b &= 2\sin \frac{a +b}{2} \cos \frac{a-b}{2} \\
    \sin a - \sin b &= 2\sin \frac{a - b}{2} \cos \frac{a + b}{2} \\
    \cos a + \cos b &= 2\cos \frac{a + b}{2} \cos \frac{a-b}{2} \\
    \cos a - \cos b &= -2\sin \frac{a+b}{2} \sin \frac{a-b}{2}
    \end{align}

    注意:这几个公式熟练记忆,以后经常要用
    
    \end{flushleft}


\section{判断题}
    \begin{flushleft}

    1.设$\lim \limits _{x \rightarrow a} f(x) = A,\lim \limits _{u \rightarrow A} g(u) = B \Rightarrow \lim \limits _{x \rightarrow a} g(f(x)) = B \quad(\quad)$

    反例:\[
    u=f(x)=x \sin \frac{1}{x}, \quad y=g(u)=    
    \left\{
    \begin{array}{ll}
    {0} & {u=0} \\ 
    {1} & {u \neq 0}
    \end{array}\right.
    \]

    2.设$\lim \limits _{x \rightarrow a} f(x) = A, \lim \limits _{u \rightarrow A} g(u) = B,\text{且存在} U^{o}(a),\text{使得在}U^{o}(a)\text{内} f(x) \neq A,$则能推出 $\lim \limits _{x \rightarrow a} g(f(x)) = B \quad(\quad)$

    \begin{proof}
    由$\lim \limits _{u \rightarrow A} g(u) = B$知,对任何的$\varepsilon > 0,$存在$\eta > 0$,使得当$0 < \vert u - A \vert < \eta$时,$\vert g(u) - B \vert < \varepsilon,$又由$\lim \limits _{x \rightarrow a} f(x) = A,$对上面的$\eta$,存在$\delta > 0$,使得当$0 < \vert x - a \vert < \delta$时,有$\vert f(x) - A \vert< \eta$.由于$f(x) \neq A$,所以当$0 < \vert x - a \vert < \delta$时,$0 < \vert f(x) - A \vert < \eta $,从而$\vert g(f(x)) - B \vert < \varepsilon$,即$\lim \limits _{x \rightarrow a}g(f(x)) = B.$
    \end{proof}

    \end{flushleft}

\section{证明题}
    \begin{flushleft}
    
    {\bfseries 归纳法证明极限题}

    1.设$0<c<1,a_1 = \frac{c}{2},a_{n+1}=\frac{c}{2}+\frac{a_n^2}{2},$证明:$\{a_n\}$收敛,并求其极限
    \begin{proof} % 宏包amsthm
    这类题一般可以这样做:首先解方程得到一个上界或下界,然后归纳法证明$\{a_n\}$确实满足这个范围
    然后根据$\{a_n\}$的范围再去用归纳法求$\{a_n\}$的单调性
    最后单调有界必有极限
    \end{proof}
    \par
    \par
    答案:极限$\lim \limits _{n \rightarrow \infty} a_n = 1 - \sqrt{1-c}$
    \par
    说明:有同学问为什么极限不取$\lim \limits _{n \rightarrow \infty}1+\sqrt{1+c}$,因为这里用归纳假设很容易看出,$a_n < 1$,所以其实一开始你去假设$0 < a_n < 1$也是可以做的

    \end{flushleft}


\end{document}

