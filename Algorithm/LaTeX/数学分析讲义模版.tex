\documentclass[12pt, a4paper]{ctexart}
\usepackage{fancyhdr}
\usepackage{graphicx}
\usepackage{harpoon}
\usepackage{ctex}
\usepackage{amsmath}
\pagestyle{plain}
\usepackage{mathrsfs}
\usepackage{amssymb}
\usepackage{amsthm}
% 页面边距设置
\usepackage[a4paper,left=2.5cm,right=2.5cm,top=2.5cm,bottom=2.5cm]{geometry} % paperwidth=11cm,scale=0.8
\usepackage{setspace} 
\renewcommand{\baselinestretch}{1.5} % 设置行间距的大小


\newtheorem{define}{定义} %在导言区使用,定义环境名

\title{数学分析辅导讲义}
\date{日期 2019/11/15}
\author{Wilson79}
  

\begin{document}
\maketitle{}

\section{计算题}

    \begin{flushleft} % 设置左对齐
    {\bfseries 等价无穷小}

    1.求极限$$\lim \limits _{x \rightarrow 0} \frac {\sin{4x}}{\sqrt{x+1}-1}$$

    解:有理化或等价替换($\sqrt[k]{x+1}\sim 1+\frac{1}{k}x$)\\
    答案:8
    
    2.求极限$$\lim \limits _{\alpha \rightarrow \beta} \frac{e^\alpha - e^\beta}{\alpha - \beta}$$
    
    解:换元$t = \alpha - \beta$
    \\答案:$e^{\beta}$

    3.求极限\[\lim \limits _{n \rightarrow \infty}\frac{{n^2\sin {\frac{1}{n}} \sqrt[n]{2}}(1-\cos \frac{1}{n^2})} {\cos \frac{1}{n}(\sqrt{n^2+1}-n)(\frac{1}{n}+1)}\]

    解:取极限时如果结果是非零常数的项可以直接取这个值,如果分子或分母出现0,则这一项不能轻易取值,要考虑等价无穷小

    答案:0

    4.求极限\[\lim \limits _{x \rightarrow 0} \frac{(3+2 \sin x)^x - 3^x}{tan ^2x}
    \]

    解:不是$1^\infty$型,两个指数形式的相减,往往需要提出公因数,这题类似上面换元的题目,这里需要用到$a^b=e^{blna}$

    答案:

    \[
    \begin{aligned}
    \lim \limits _{x \rightarrow 0} \frac{(3+2 \sin x)^x - 3^x}{tan ^2x} &= \lim \limits _{x \rightarrow 0} {\frac{3^x\left((1+\frac{2}{3}\sin x)^x-1\right)} {\tan^2 x}} \\
    &=\lim \limits _{x \rightarrow 0} \frac{3^x(e^{xln(1+\frac{2}{3}\sin x) }-1)}{\tan^2x} 
    &=\lim \limits _{x \rightarrow 0} \frac{3^xxln(1+\frac{2}{3} \sin x)}{\tan^2 x}
    & = \frac{2}{3}
    \end{aligned}
    \]

    {\bfseries 多项式求极限模型}
    
    看次数最高的项
    \[
    \lim _{n \rightarrow \infty} \frac{a_{m} n^{m}+a_{m-1} n^{m-1}+\cdots+a_{1} n+a_{0}}{b_{k} n^{k}+b_{k-1} n^{k-1}+\cdots+b_{1} n+b_{0}}=\left\{\begin{array}{ll}{\frac{a_{m}}{b_{m}},} & {k=m} \\ {0,} & {k>m}\end{array}\right.
    \] % \{是特殊字符,\left是定界符

    5.
    \[\lim \limits _{n \rightarrow \infty} \frac{3n^3+n}{2n^3+n^2}=\frac{3}{2}\]
    \[\lim \limits _{x \rightarrow +\infty} \frac{\sqrt{x+\sqrt{x+\sqrt{x}}}}{\sqrt{2x+1}}=\frac{\sqrt{2}}{2}\]
    \[\lim \limits _{x \rightarrow 0+} \frac{\sqrt[3]{x^5+x^3+x}+2x^2}{x^{\frac{5}{3}}}=\infty\]
    如果x趋于0,建议先用换元$n=\frac{1}{x}$

    {\bfseries 放缩法}

    6.求极限
    \[\lim \limits _{n \rightarrow \infty} \sqrt[n]{1-\frac{1}{n}}\]

    7.求极限\[
    \lim \limits _{n \rightarrow \infty} \sqrt[n]{3^n + n^2}
    \]


    8.求极限\[
    \lim \limits _{n \rightarrow \infty} \left(\frac{1}{\sqrt{n^2+1}} + \frac{1}{\sqrt{n^2+2}} + \cdots + \frac{1}{\sqrt{n^2+n}} \right)
    \]
    \[\frac{1}{\sqrt{n^2+n}} \leq \frac{1}{ \sqrt{n^2+k} } \leq \frac{1}{ \sqrt{n^2+1} }\]

    9. 求极限
    $$
    \lim _{x \rightarrow 0^{+}}\left(1+x+x^{2}\right)^{\sin \frac{1}{x}}
    $$

    解:
    \[
    \left(1+x+x^{2}\right)^{-1} \leqslant\left(1+x+x^{2}\right)^{\sin \frac{1}{x}} \leqslant\left(1+x+x^{2}\right)^{1}
    \]

    由迫敛性,得$$\lim _{x \rightarrow 0^{+}} (1+x+x^2)^{\sin \frac{1}{x}} = 1$$

    或者\[
    (1+x+x^2)^{\frac{1}{x+x^2}(x+x^2)\sin \frac{1}{x}}
    \]

    {\bfseries 和差化积-积化和差公式}

  
    \begin{align}
    \sin a + \sin b &= 2\sin \frac{a +b}{2} \cos \frac{a-b}{2} \\
    \sin a - \sin b &= 2\sin \frac{a - b}{2} \cos \frac{a + b}{2} \\
    \cos a + \cos b &= 2\cos \frac{a + b}{2} \cos \frac{a-b}{2} \\
    \cos a - \cos b &= -2\sin \frac{a+b}{2} \sin \frac{a-b}{2}
    \end{align}

    注意:这几个公式熟练记忆,以后经常要用
    

    

    10.求极限 \[\lim \limits _{x \rightarrow +\infty} (\sin\sqrt{1+x} - \sin \sqrt{x})\] 

    答案:0

    11.化简\[\sum_{k=1}^n \cos {kx} \qquad x \in (0, 2\pi)\] 

    解:
    \[
    \begin{aligned}
    2\sin \frac{x}{2}\left(\frac{1}{2} + \sum_{k=1}^n \cos {kx}   \right) = &\sin \frac{x}{2} + \left(\sin \frac{3}{2}x - \sin \frac{x}{2}\right) + \cdots \\
    & + \left[ \sin(n + \frac{1}{2}) x - \sin(n - \frac{1}{2})x \right] \\
    = &\sin(n +\frac{1}{2})x
    \end{aligned}
    \]




    \end{flushleft}


\section{判断题}
    \begin{flushleft}
    1.若$\lim \limits _{x \rightarrow 1} f(x)g(x) = 0\text{且}\forall x \in (0,2),g(x)>0,\text{则} \lim \limits _{x \rightarrow 1} f(x) = 0 \quad(\quad)$

    反例:\[
    f(x) =     
    \left\{
    \begin{array}{ll}
    {1} & {x \neq 1} \\ 
    {0} & {x = 1}
    \end{array}\right.
    g(x) =     
    \left\{
    \begin{array}{ll}
    {(x-1)^2} & {x \neq 1} \\ 
    {1} & {x = 1}
    \end{array}\right. 
    f(x)g(x)=    
    \left\{
    \begin{array}{ll}
    {(x-1)^2} & {x \neq 1} \\ 
    {0} & {x = 1}
    \end{array}\right.
    \]
    


    2.设$\lim \limits _{x \rightarrow a} f(x) = A,\lim \limits _{u \rightarrow A} g(u) = B \Rightarrow \lim \limits _{x \rightarrow a} g(f(x)) = B \quad(\quad)$

    反例:\[
    u=f(x)=x \sin \frac{1}{x}, \quad y=g(u)=    
    \left\{
    \begin{array}{ll}
    {0} & {u=0} \\ 
    {1} & {u \neq 0}
    \end{array}\right.
    \]

    3.设$\lim \limits _{x \rightarrow a} f(x) = A, \lim \limits _{u \rightarrow A} g(u) = B,\text{且存在} U^{o}(a),\text{使得在}U^{o}(a)\text{内} f(x) \neq A,$则能推出 $\lim \limits _{x \rightarrow a} g(f(x)) = B \quad(\quad)$

    \begin{proof}
    由$\lim \limits _{u \rightarrow A} g(u) = B$知,对任何的$\varepsilon > 0,$存在$\eta > 0$,使得当$0 < \vert u - A \vert < \eta$时,$\vert g(u) - B \vert < \varepsilon,$又由$\lim \limits _{x \rightarrow a} f(x) = A,$对上面的$\eta$,存在$\delta > 0$,使得当$0 < \vert x - a \vert < \delta$时,有$\vert f(x) - A \vert< \eta$.由于$f(x) \neq A$,所以当$0 < \vert x - a \vert < \delta$时,$0 < \vert f(x) - A \vert < \eta $,从而$\vert g(f(x)) - B \vert < \varepsilon$,即$\lim \limits _{x \rightarrow a}g(f(x)) = B.$
    \end{proof}

    \end{flushleft}

\section{证明题}
    \begin{flushleft}
    
    {\bfseries 归纳法证明极限题}

    1.设$0<c<1,a_1 = \frac{c}{2},a_{n+1}=\frac{c}{2}+\frac{a_n^2}{2},$证明:$\{a_n\}$收敛,并求其极限
    \begin{proof} % 宏包amsthm
    这类题一般可以这样做:首先解方程得到一个上界或下界,然后归纳法证明$\{a_n\}$确实满足这个范围
    然后根据$\{a_n\}$的范围再去用归纳法求$\{a_n\}$的单调性
    最后单调有界必有极限
    \end{proof}
    \par
    \par
    答案:极限$\lim \limits _{n \rightarrow \infty} a_n = 1 - \sqrt{1-c}$
    \par
    说明:有同学问为什么极限不取$\lim \limits _{n \rightarrow \infty}1+\sqrt{1+c}$,因为这里用归纳假设很容易看出,$a_n < 1$,所以其实一开始你去假设$0 < a_n < 1$也是可以做的


    2.用类似地方法可以证明$a_n = \sqrt{2 + a_{n-1}}\qquad a_1=\sqrt{2}$

    提示:假设$a_{n-1} \leq a_{n}$,有$a_n = \sqrt{2+a_{n-1}} \leq \sqrt{2+a_{n} }= a_{n+1}$
    \end{flushleft}


\end{document}

