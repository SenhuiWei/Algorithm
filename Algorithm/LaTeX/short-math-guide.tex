% This is file `short-math-guide.tex'.
%
% Copyright 1995-2017
% American Mathematical Society
% 
% American Mathematical Society
% Technical Support
% Publications Technical Group
% 201 Charles Street
% Providence, RI 02904
% USA
% tel: (401) 455-4080
%      (800) 321-4267 (USA and Canada only)
% fax: (401) 331-3842
% email: tech-support@ams.org
% 
% This work may be distributed and/or modified under the
% conditions of the LaTeX Project Public License, either
% version 1.3c of this license or (at your option) any
% later version.  The latest version of this license is in
%    http://www.latex-project.org/lppl.txt
% and version 1.3c or later is part of all distributions of LaTeX 
% version 2005/12/01 or later.
% 
% This file has the LPPL maintenance status "maintained".
% 
% The Current Maintainer of this work is the American Mathematical
% Society.
% 
% Note: When updating, don't forget to update \smgversion, below

\begin{filecontents}{mathdoc.sty}
\NeedsTeXFormat{LaTeX2e}[1995/12/01]
\ProvidesPackage{mathdoc}[2017/12/22 v2.0]

\providecommand{\lat}[1]{\protect\LaTeX{}}
\providecommand{\mdash}{\textemdash}
\providecommand{\ndash}{\textendash}
\newcommand{\ntt}{\fontseries{m}\fontshape{n}\ttfamily}
\newcommand{\mntt}[1]{\mbox{\ntt #1}}
\providecommand{\pkg}[1]{\mntt{#1}}
\providecommand{\cls}[1]{\mntt{#1}}
\providecommand{\opt}[1]{\mntt{#1}}
\providecommand{\fn}[1]{\mntt{#1}}
\providecommand{\env}[1]{\mntt{#1}}
\chardef\bslash=`\\
\let\charhack=\char
\providecommand{\cn}[1]{\mntt{\bslash\charhack`#1}}
\newcommand{\hycn}{\protect\cn}
\@ifpackageloaded{hyperref}{%
  \def\mntt#1{\texttt{\upshape #1}}%
  \def\hycn#1{\texttt{\upshape\protect\bslash#1}}%
  \begingroup \lccode`\.=`\\ \lowercase{\endgroup
  \def\bslash{\texorpdfstring{.}{\textbackslash}}
  }
}{}
\providecommand{\ncn}{\cn}
\providecommand{\qq}[1]{\textquotedblleft#1\textquotedblright}

\oddsidemargin=0pt \evensidemargin=0pt

\let\Huge\Large \let\huge\Large \let\LARGE\large \let\Large\large

\def\section{\@startsection{section}{1}%
  \z@{9pt plus12pt}{1.5ex}%
  {\bfseries\global\@afterindentfalse}}

\def\subsection{\@startsection{subsection}{2}%
  \z@{9pt plus12pt}{-.5em}%
  {\scshape}}

\def\@seccntformat#1{\csname the#1\endcsname.\enskip}

%    Durn it, looks like a LaTeX kernel bug: consecutive "run-in" section
%    titles should not allow a page break between them.
\def\@startsection #1#2#3#4#5#6{%
  \if@noskipsec \leavevmode\par\nobreak\vskip\medskipamount\@nobreaktrue\fi
  \par
  \@tempskipa#4\relax \@afterindenttrue
  \ifdim\@tempskipa<\z@ \@tempskipa-\@tempskipa \@afterindentfalse \fi
  \if@nobreak \everypar{}%
  \else \addpenalty\@secpenalty \addvspace\@tempskipa
  \fi
  \@ifstar{\@ssect{#3}{#4}{#5}{#6}}%
          {\@dblarg{\@sect{#1}{#2}{#3}{#4}{#5}{#6}}}%
}

%    Redefine \@sect to add period after run-in headings
\def\@sect#1#2#3#4#5#6[#7]#8{%
  \ifnum #2>\c@secnumdepth
    \let\@svsec\@empty
  \else
    \refstepcounter{#1}%
    \protected@edef\@svsec{\@seccntformat{#1}\relax}%
  \fi
  \@tempskipa #5\relax
  \ifdim \@tempskipa>\z@
    \begingroup
      #6{%
        \@hangfrom{\hskip #3\relax\@svsec}%
          \interlinepenalty \@M #8\@@par}%
    \endgroup
    \csname #1mark\endcsname{#7}%
    \addcontentsline{toc}{#1}{%
      \ifnum #2>\c@secnumdepth \else
        \protect\numberline{\csname the#1\endcsname}%
      \fi
      #7}%
  \else
    \def\@svsechd{%
      #6{\hskip #3\relax
      \@svsec #8\@addpunct.}%
      \csname #1mark\endcsname{#7}%
      \addcontentsline{toc}{#1}{%
        \ifnum #2>\c@secnumdepth \else
          \protect\numberline{\csname the#1\endcsname}%
        \fi
        #7}}%
  \fi
  \@xsect{#5}}
%    From amsart, allows \nopunct to omit unwanted periods
\def\@addpunct#1{%
  \relax\ifhmode
    \ifnum\spacefactor>\@m \else#1\fi
  \fi}
\def\nopunct{\texorpdfstring{\spacefactor 1007 }{}}

%    Workaround to avoid overprinting problem with first contents entry
\let\zz@l@section\l@section
\def\l@section{\if@noskipsec \mbox{}\par\fi\zz@l@section}

\RequirePackage{keyval}\relax

\define@key{symlist}{adjustwidth}{\advance\wdadjust#1\relax}
\define@key{symlist}{adjustheight}{\htadjust#1\relax}
\define@key{symlist}{adjustcols}{\coladjust#1\relax}

\providecommand{\newcolumn}{\vfil\break}

\newenvironment{symlist}[1][]{%
  \if@noskipsec\ifvmode\nobreak\fi\leavevmode\fi
  \par
  \nobreak
  \wdadjust=1.5em\relax
  \gdef\containsMSABM{TF}%
  \setkeys{symlist}{#1}%
    \setbox\z@\vbox\bgroup
      \advance\baselineskip \z@ plus2pt\relax
      % work around splittopskip discrepancy
      \null \penalty-\@M
}{%
  \par\egroup
  \splitlist
}

\let\debugit\relax
\def\boxo{%
  \typeout{Box0: \the\wd0x\the\ht0+\the\dp0, splitting to
    \the\dimen@\space for \number\cols\space columns}%
}

\newdimen\shiftlistright
\shiftlistright=20pt
\def\splitlist{%
  \begingroup
  \textwidth=400pt % fudgit
  \dimen@=\wd0 \advance\dimen@\wdadjust \advance\dimen@\colsep
  \cols\textwidth \advance\cols\colsep
  \divide\cols\dimen@ \advance\cols\coladjust
  \ifnum\cols<\z@ \cols=\@ne \fi
  \dimen@\ht0 \divide\dimen@\cols
  \advance\dimen@\baselineskip \divide\dimen@\baselineskip
  \multiply\dimen@\baselineskip
  \splittopskip10pt\relax \splitmaxdepth\maxdepth
  \advance\dimen@\splittopskip \advance\dimen@-\baselineskip
  \advance\dimen@\htadjust
  \vbadness\@M % ignore underfull vbox messages
\debugit
  \def\do{%
    \advance\curcol 1
    \setbox2=\vsplit0 to\dimen@
    \ifnum\curcol=\cols \vtop \else \vtop to\dimen@\fi
      {\unvbox 2 }\hskip\colsep
    \ifdim\ht0>\z@ \expandafter\do\fi
  }%
  \setbox2=\vsplit0 to\baselineskip % discard empty top box
  \hbox to\textwidth{\hskip\shiftlistright\quad\curcol=0 \do\unskip\hfil}%
  % cancel a large prevdepth
  \nobreak\nointerlineskip\hbox{}%
  \endgroup
  \global\shiftlistright20pt
}

\def\dosymbol#1{\csname do#1\endcsname}
\newcommand{\ttfont}{%
  \normalfont\ttfamily
  \global\expandafter\let\expandafter\ttfont\the\font
}

\newdimen\wdadjust \newdimen\htadjust
\newskip\colsep \newcount\cols \newcount\curcol \newcount\coladjust
\colsep=10pt plus1fil minus2pt

\newbox\symstrut
\AtBeginDocument{\setbox\symstrut\hbox{\vrule height8.5pt depth3.5pt width0pt}}

\newcommand{\symbox}[2]{%
  \hbox{\llap{\unhcopy\symstrut $ #1 $ }\ttfont#2}%
}

\edef\symnote#1{%
  \noexpand\expandafter\noexpand\symnoteA
  \noexpand\meaning#1\relax\string h?"00\noexpand\@nil
}

\begingroup\edef\x#1h{\endgroup #1\string h}\x
\def\symnoteA#1h#2#3"#4#5#6\@nil{%
  \if c#2\relax
    \expandafter\ifx\csname#2#3\endcsname\char
      \ifcase#5\relax \or\or\or\or
        \symnoteB[to 0pt]{a}\or\symnoteB[to 0pt]{b}\else\fi
    \fi
  \fi
}

\newcommand{\symnoteB}[2][]{%
  \hbox #1{%
    \raise0.9ex\hbox{\fontfamily{cmr}\fontseries{m}\scriptsize#2}\hss
  }%
  \gdef\containsMSABM{TT}%
}


\newcommand{\printSymbol}[1]{%
  \hbox{\llap{\unhcopy\symstrut $#1$ }\ttfont\string#1%
    \symnote#1%
  }%
}

\newcommand{\printBig}[1]{\hbox{\llap{$\Big#1$ }\ttfont\string#1}}

% Arg 2 is something like "kernel" or "amssymb"; but
% if we want to distinguish msam or msbm we have to look more closely.

\newcommand{\doVar}[2]{\expandafter\printSymbol\csname#1\endcsname}

\newcommand{\doVarc}[2]{%
  \hbox{\llap{\unhcopy\symstrut$#1$ }\ttfont#1}%
}

\newcommand{\doDeL}[2]{\expandafter\printBig\csname#1\endcsname}%

\newcommand{\doDeLc}[2]{%
  \hbox{\llap{$\Big#1$ }\ttfont#1}%
}

\let\doOrd\doVar
\let\doBin\doVar
\let\doRel\doVar
\let\doPun\doVar
\let\doOrdc\doVarc
\let\doBinc\doVarc
\let\doRelc\doVarc
\let\doPunc\doVarc
\let\doInn\doVar
\let\doCOi\doVar
\let\doCOs\doVar
\let\doDeR\doDeL
\let\doDeB\doDeL
\let\doDeBc\doDeLc
\let\doDeA\doDeL

\newcommand{\doDeLR}[3]{%
  \hbox{\llap{$
    \expandafter\Bigl\csname#1\endcsname\,
    \expandafter\Bigr\csname#2\endcsname$ }%
    \ttfont\bslash#1 \bslash#2}%
}

\newcommand{\doDeLRc}[3]{%
  \hbox{\llap{$\Bigl#1\,\Bigr#2$ }\ttfont\string#1 \string#2}%
}

\newcommand{\doOrdx}[2]{%
  \hbox{\llap{\unhcopy\symstrut$#1$ }\ttfont\string#1}%
}

\let\doOpn\doVar

\newcommand{\doFsw}[2]{%
  \hbox{\kern-\parindent $\csname#1\endcsname{R}$\space
    \ttfont\string#1\string{R\string}}%
}

\newcommand{\doAcc}[2]{%
  \hbox{\llap{\unhcopy\symstrut$\csname#1\endcsname{x}$ }%
    \ttfont\bslash#1\string{x\string}}%
}

\newcommand{\doAccw}[2]{%
  \hbox{\llap{\unhcopy\symstrut$\csname#1\endcsname{xxx}$ }%
    \ttfont\bslash#1\string{xxx\string}}%
}

\newcommand{\alias}[1]{$\csname#1\endcsname$ \cn{#1}}

\newcommand{\symrow}[1]{%
  {#1}& \mathbf{#1}& \mathrm{#1}& \mathsf{#1}& \mathit{#1}& \mathcal{#1}&
  \mathbb{#1}& \mathfrak{#1}%
}
%  \mathscr{#1}&  \mathbb{#1}& \mathfrak{#1}%
%% rsfs has nothing in some of the example slots, gets ! Missing character

\newenvironment{eqxample}{%
  \par\addvspace\medskipamount
  \noindent\begin{minipage}{.5\columnwidth}%
  \def\producing{\end{minipage}\begin{minipage}{.5\columnwidth}%
    \hbox\bgroup\kern-.2pt\vrule width.2pt\vbox\bgroup\parindent0pt\relax
%    The 3pt is to cancel the -\lineskip from \displ@y
    \abovedisplayskip3pt \abovedisplayshortskip\abovedisplayskip
    \belowdisplayskip0pt \belowdisplayshortskip\belowdisplayskip
    \noindent}
}{%
  \par
%    Ensure that a lonely \[\] structure doesn't take up width less than
%    \hsize.
  \hrule height0pt width\hsize
  \egroup\vrule width.2pt\kern-.2pt\egroup
  \end{minipage}%
  \par\addvspace\medskipamount
}

\newcommand{\noteslabel}[1]{\hskip\labelsep\textit{#1\unskip}}

\newcommand{\singlenote}{\item[\textit{Note.}]}
\newcommand{\synonyms}{\item[\textit{Synonyms\/}:]}

\newenvironment{notes}{%
  \begin{list}{Note \arabic{enumiv}.}{%
    \usecounter{enumiv}%
    \footnotesize
    \setlength{\leftmargin}{0pt}%
    \setlength{\labelwidth}{0pt}%
    \setlength{\topsep}{\medskipamount}%
    \renewcommand{\makelabel}{\noteslabel}%
  }%
  \if\containsMSABM %
    \item Labels \symnoteB[to 1.3em]{a,b} indicate \pkg{amssymb}
      package, font \fn{msam} or~\fn{msbm}.%
  \fi
  \gdef\containsMSABM{TF}%
}{%
  \end{list}%
}

\newcommand{\ma}[1]{%
  \string{{\normalfont\itshape #1}\string}\penalty9999 \ignorespaces}

\newenvironment{cmdspec}[1][\linewidth]{%
  \begin{center}\begin{minipage}{#1}%
  \raggedright \normalfont\ttfamily \exhyphenpenalty10000
}{%
  \end{minipage}\end{center}%
}

%    l2h has this screwed up somehow? [mjd,1999/11/05]
\newenvironment{tex2html_preform}{}{}

\providecommand{\strong}{\textbf}

\@ifundefined{ht@url}{
  \newcommand\htlink{\href}
  \newenvironment{makeimage}{}{}
}{
  \renewcommand\htlink[2]{#1\htlinkfootnote{\ht@url{#2}}}
  \newcommand{\htlinkfootnote}[1]{}
}

\providecommand{\url}{\texttt}
\endinput
\end{filecontents}

%%%%%%%%%%%%%%%%%%%%%%%%%%%%%%%%%%%%%%%%%%%%%%%%%%%%%%%%%%%%%%%%%%%%%%%%

\errorcontextlines=99

\documentclass{article}

\pagestyle{myheadings}
%\usepackage{hthtml}
\usepackage{amsmath}
%\usepackage[cmex10]{amsmath}
\usepackage{amssymb}
\usepackage{mathrsfs}
%\usepackage[mathcal]{euscript}
\usepackage{euscript}
%\usepackage{mathtime}
%\usepackage{stmaryrd}

\numberwithin{equation}{section}

%% at the moment, using either url or hyperref crashes
%% get the content correct before debugging that problem
%\usepackage{url}
\let\url\texttt
\usepackage[breaklinks,colorlinks]{hyperref}
\usepackage{xcolor}
\definecolor{hycitecolor}{rgb}{0,0.65,0}

\usepackage{mathdoc}

\hoffset=-1 true in
\voffset=-1 true in
\topmargin=0.75 true in %% \oddsidemargin\topmargin
%\textwidth=210 true mm % A4
\textwidth=139 true mm
\textheight=11 true in \advance\textheight-2\topmargin
\headheight=7pt \advance\textheight-\headheight
\headsep=11pt \advance\textheight-\headsep

\oddsidemargin=\paperwidth
\advance\oddsidemargin-\textwidth
\oddsidemargin=.5\oddsidemargin
\evensidemargin=\oddsidemargin

\hfuzz=14pt % suppress uninteresting warnings

\providecommand{\abs}[1]{\lvert#1\rvert}

\newcommand{\colhead}[1]{%
  \textbf{\begin{tabular}[b]{@{}l@{}}#1\end{tabular}}%
}

\newcommand{\secref}[1]{Section~\ref{#1}}
\newcommand{\tabref}[1]{Table~\ref{#1}}

\newcommand{\begend}[1]{%
  \cn{begin}\texttt{\symbol{123}#1\symbol{125}}%
  \ \dots\ \cn{end}\texttt{\symbol{123}#1\symbol{125}}%
}

\newcommand{\dbldollars}{\texttt{\$\$} \dots\ \texttt{\$\$}}

\newenvironment{lstack}[1][t]{%
  \begin{tabular}[#1]{@{}l@{}}%
}{%
  \end{tabular}
}  

\newenvironment{cstack}[1][t]{%
  \begin{tabular}[#1]{@{}c@{}}%
}{%
  \end{tabular}
}  

\newenvironment{llstack}[1][t]{%
  \begin{tabular}[#1]{@{}ll@{}}%
}{%
  \end{tabular}
}  

%%\newcommand{\lspx}{\mbox{\rule{5pt}{.6pt}\rule{.6pt}{6pt}}}
%%\newcommand{\rspx}{\mbox{\rule[-1pt]{.6pt}{7pt}%
%%  \rule[-1pt]{5pt}{.6pt}}}
%%\newcommand{\lspx}{\mathord{\otimes}}
%%\newcommand{\rspx}{\mathord{\odot}}
\newcommand{\lspx}{3}
\newcommand{\rspx}{4}
\newcommand{\spx}[1]{$\lspx #1\rspx$}

\DeclareMathOperator{\rank}{rank}
\DeclareMathOperator{\esssup}{ess\,sup}

\newcommand{\dotsref}{\leavevmode\unskip\space
  (see Section~\ref{dots})}
\newcommand{\vertref}{\leavevmode\unskip\space
  (see Section~\ref{verts})}

\providecommand{\pdfinfo}[1]{}

%%%%%%%%%%%%%%%%%%%%%%%%%%%%%%%%%%%%%%%%%%%%%%%%%%%%%%%%%%%%%%%%%%%%%%%%

\begin{document}
\title{Short Math Guide for \LaTeX{}}
\newcommand{\smgversion}{\textup{2.0 (2017/12/22)}}
\markboth{Short Math Guide for \LaTeX{}, version \protect\smgversion}
         {Short Math Guide for \LaTeX{}, version \protect\smgversion}
\author{Michael Downes, updated by Barbara Beeton}
\date{American Mathematical Society}

\pdfinfo{
  /Title (Short Math Guide for LaTeX)
  /Author (Michael Downes, updated by Barbara Beeton)
  /Subject (Short Math Guide for LaTeX)
  /Keywords (LaTeX,amsmath,amsfonts,amssymb,equation,math,formula)
}

\maketitle
\begin{center}
Version \smgversion, currently available from a link at
\mbox{\texttt{https://www.ams.org/tex/amslatex}}
\end{center}
%%\begin{rawhtml}
%%<p>
%%<a href="../short-math-guide.pdf">Download a PDF version of this documentation.</a>
%%</p>
%%<hr/>
%%\end{rawhtml}

\setcounter{tocdepth}{3}
\tableofcontents

\vspace{\fill}
\section*{Acknowledgments and plans for future work}
Thanks to all who contributed suggestions, assistance and encouragement.
Special thanks to David Carlisle for repairing unruly macros and to
Jennifer Wright Sharp for applying consistent editing in AMS style.

\smallskip\noindent
Plans for a future edition include addition of an index.

\smallskip\noindent
Reports concerning errors and suggestions for improvement should be
sent to\\[2pt]
\hspace*{\fill}
\href{mailto:tech-support@ams.org}{\texttt{tech-support@ams.org}}\,.
\hspace{\fill}\null

\newpage

%%%%%%%%%%%%%%%%%%%%%%%%%%%%%%%%%%%%%%%%%%%%%%%%%%%%%%%%%%%%%%%%%%%%%%%%

\section{Introduction}

This is a concise summary of recommended features in \LaTeX{} and a
couple of extension packages for \textbf{writing math formulas}. Readers
needing greater depth of detail are referred to the sources listed in
the bibliography, especially \cite{lamport}, \cite{amsldoc}, and
\cite{fntguide}. A certain amount of familiarity with standard \LaTeX{}
terminology is assumed; if your memory needs refreshing on the \LaTeX{}
meaning of \emph{command}, \emph{optional argument}, \emph{environment},
\emph{package}, and so forth, see \cite{lamport}.

Most of the features described here are available to you if you use
\LaTeX{} with two extension packages published by the American Mathematical
Society: \pkg{amssymb} and \pkg{amsmath}. Thus, the source file for this
document begins with
\begin{verbatim}
\documentclass{article}
\usepackage{amssymb,amsmath}
\end{verbatim}
The \pkg{amssymb} package might be omissible for documents whose math
symbol usage is relatively modest; in \secref{mathsymbols}, the symbols
that require \pkg{amssymb} are marked with \textsuperscript{a} or
\textsuperscript{b} (font \fn{msam} or \fn{msbm}).  In \secref{alpha-digit},
a few additional fonts are included; the necessary packages are identified
there.

Many noteworthy features found in other packages are not covered here;
see \secref{other-packages}. Regarding math symbols, please note
especially that the list given here is not intended to be comprehensive,
but to illustrate such symbols as users will normally find already
present in their \lat/ system and usable without installing any
additional fonts or doing other setup work.

If you have a need for a symbol not shown here, you will probably want
to consult \emph{The Comprehensive \LaTeX{} Symbol List}~\cite{comprehensive}.
%\[\url{http://www.ctan.org/tex-archive/info/symbols/comprehensive/}\]
If your \lat/ installation is based on \TeX\,Live, and includes documentation,
the list can also be accessed by typing \texttt{texdoc comprehensive} at a
system prompt.

\begin{table}[p]
\caption[]{Multiline equations and equation groups\\
 \phantom{Table 1:} (vertical lines indicate nominal margins).}
\label{displays}
\bigskip
\begin{makeimage}
\begin{minipage}{\textwidth}
\begin{eqxample}
\begin{verbatim}
\begin{equation}\label{xx}
\begin{split}
a& =b+c-d\\
 & \quad +e-f\\
 & =g+h\\
 & =i
\end{split}
\end{equation}
\end{verbatim}
\producing
\begin{equation}\label{xx}
\begin{split}
a& =b+c-d\\
 & \quad +e-f\\
 & =g+h\\
 & =i
\end{split}
\end{equation}
\end{eqxample}

\begin{eqxample}
\begin{verbatim}
\begin{multline}
a+b+c+d+e+f\\
+i+j+k+l+m+n\\
+o+p+q+r+s
\end{multline}
\end{verbatim}
\producing
\begin{multline}
a+b+c+d+e+f\\
+i+j+k+l+m+n\\
+o+p+q+r+s
\end{multline}
\end{eqxample}

\begin{eqxample}
\begin{verbatim}
\begin{gather}
a_1=b_1+c_1\\
a_2=b_2+c_2-d_2+e_2
\end{gather}
\end{verbatim}
\producing
\begin{gather}
a_1=b_1+c_1\\
a_2=b_2+c_2-d_2+e_2
\end{gather}
\end{eqxample}

\begin{eqxample}
\begin{verbatim}
\begin{align}
a_1& =b_1+c_1\\
a_2& =b_2+c_2-d_2+e_2
\end{align}
\end{verbatim}
\producing
\begin{align}
a_1& =b_1+c_1\\
a_2& =b_2+c_2-d_2+e_2
\end{align}
\end{eqxample}

\begin{eqxample}
\begin{verbatim}
\begin{align}
a_{11}& =b_{11}&
  a_{12}& =b_{12}\\
a_{21}& =b_{21}&
  a_{22}& =b_{22}+c_{22}
\end{align}
\end{verbatim}
\producing
\begin{align}
a_{11}& =b_{11}&
  a_{12}& =b_{12}\\
a_{21}& =b_{21}&
  a_{22}& =b_{22}+c_{22}
\end{align}
\end{eqxample}

\begin{eqxample}
\begin{verbatim}
\begin{alignat}{2}
a_1& =b_1+c_1&      &+e_1-f_1\\
a_2& =b_2+c_2&{}-d_2&+e_2
\end{alignat}
\end{verbatim}
\producing
\begin{alignat}{2}
a_1& =b_1+c_1&      &+e_1-f_1\\
a_2& =b_2+c_2&{}-d_2&+e_2
\end{alignat}
\end{eqxample}

\begin{eqxample}
\begin{verbatim}
\begin{flalign}
a_{11}& =b_{11}&
  a_{12}& =b_{12}\\
a_{21}& =b_{21}&
  a_{22}& =b_{22}+c_{22}
\end{flalign}
\end{verbatim}
\producing
\begin{flalign}
a_{11}& =b_{11}&
  a_{12}& =b_{12}\\
a_{21}& =b_{21}&
  a_{22}& =b_{22}+c_{22}
\end{flalign}
\end{eqxample}
\def\containsMSABM{TF}
\begin{notes}
\item Applying \env{*} to any primary environment will suppress the
  assignment of equation numbers.  However, \cn{tag} may be used to
  apply a visible label, and \cn{eqref} can be used to reference
  such manually tagged lines.   Use of either \env{*} or a \cn{tag}
  on a subordinate environment is an error.
\item The \env{split} environment is something of a
  special case. It is a subordinate environment that can be used as the
  contents of an \env{equation} environment or the contents of one
  \qq{line} in a multiple-equation structure such as \env{align} or
  \env{gather}.
\item The primary environments \env{gather}, \env{align} and \env{alignat}
  have subordinate ``\env{-ed}'' counterparts (\env{gathered},
  \env{aligned} and \env{alignedat}) that can be used as components of
  more complicated displays, or within in-line math.  These ``\env{-ed}''
  environments can be positioned vertically using an optional argument
  \verb+[t]+, \verb+[c]+ or~\verb+[b]+.
\item The name \env{flalign} is meant as ``full length'', not
  ``flush left'' as often mistakenly reported.  However, since a
  display occupying the full width will often begin at the left
  margin, this confusion is understandable.  The indent applied to
  \env{flalign} from both margins is set with \cn{multlinegap}.
\end{notes}
\end{minipage}
\end{makeimage}
\end{table}

%%%%%%%%%%%%%%%%%%%%%%%%%%%%%%%%%%%%%%%%%%%%%%%%%%%%%%%%%%%%%%%%%%%%%%%%

\section{Inline math formulas and displayed equations}\label{first-step}

\subsection{The fundamentals}

Entering and leaving math mode in \LaTeX{} is normally done with the
following commands and environments.%
\begin{center}
\begin{tabular}{ccc}
\colhead{inline formulas}&& \colhead{displayed equations}\\[3pt]
\cline{1-1}\cline{3-3}\noalign{\medskip}
\begin{cstack}
  \verb'$' \dots\ \verb'$'\\
  \verb'\(' \dots\ \verb'\)'
\end{cstack}%
&&
\begin{llstack}
\begin{lstack}\verb'\[...\]'\\[6pt]\end{lstack}&
  unnumbered\\
\begin{lstack}
  \verb'\begin{equation*}'\\
  \dots\\
  \verb'\end{equation*}'\\[6pt]
\end{lstack}&
  unnumbered\\
\begin{lstack}
  \verb'\begin{equation}'\\
  \dots\\
  \verb'\end{equation}'
\end{lstack}&
  \begin{lstack}automatically\\numbered\end{lstack}
\end{llstack}
\end{tabular}
\begin{notes}
%  \singlenote
\item Do not leave a blank line between text and a displayed equation.
  This allows a page break at that location, which is bad style.  It
  also causes the spacing between text and display to be incorrect,
  usually larger than it should be.
  If a visual break is desired in the input, insert a line containing
  only a \verb+%+ at the beginning.
  Leave a blank line between a display and following text only if a new
  paragraph is intended.
\item Do not group multiple display structures in the input (\verb+\[...\]+,
  \env{equation}, etc.).  Instead, use a multiline structure with
  substructures (\env{split}, \env{aligned}, etc.)\ as appropriate.
\item The alternative environments \begend{math} and\\
  \begend{displaymath} are seldom needed in practice.  Using the plain
  \TeX{} notation \dbldollars\ for displayed equations is strongly
  discouraged. Although it is not expressly forbidden in \LaTeX{}, it
  is not documented anywhere in the \LaTeX{} book as being part of the
  \LaTeX{} command set, and it interferes with the proper operation of
  various features such as the \opt{fleqn} option.
\item The \env{eqnarray} and \env{eqnarray*} environments described
  in \cite{lamport} are strongly discouraged because they produce
  inconsistent spacing of the equal signs and make no attempt to prevent
  overprinting of the equation body by the equation number.
\end{notes}
\end{center}
Environments for handling equation groups and multiline equations are
shown in \tabref{displays}.

\newpage

\subsection{Automatic numbering and cross-referencing}

To get an auto-numbered equation, use the \env{equation} environment; to
assign a label for cross-referencing, use the \cn{label} command:
\begin{verbatim}
\begin{equation}\label{reio}
...
\end{equation}
\end{verbatim}
To get a cross-reference to an auto-numbered equation, use the
\cn{eqref} command:
\begin{verbatim}
... using equations~\eqref{ax1} and~\eqref{bz2}, we
can derive ...
\end{verbatim}
The above example would produce something like
\begin{quote}
  using equations (3.2) and (3.5), we can derive
\end{quote}
In other words, \verb'\eqref{ax1}' is equivalent to \verb'(\ref{ax1})',
but the parentheses produced by \cn{eqref} are always upright.

To give your equation numbers the form \textit{m.n}
(\textit{section-number.equation-number}), use the \cn{numberwithin}
command in the preamble of your document:
\begin{verbatim}
\numberwithin{equation}{section}
\end{verbatim}
For more details on custom numbering schemes see \cite[\S 6.3,
\S C.8.4]{lamport}.

The \env{subequations} environment provides a convenient way to number
equations in a group with a subordinate numbering scheme. For example,
supposing that the current equation number is \theequation, write
\begin{verbatim}
\begin{equation}\label{first}
a=b+c
\end{equation}
some intervening text
\begin{subequations}\label{grp}
\begin{align}
a&=b+c\label{second}\\
d&=e+f+g\label{third}\\
h&=i+j\label{fourth}
\end{align}
\end{subequations}
\end{verbatim}
to get
\begin{equation}\label{first}
a=b+c
\end{equation}
some intervening text
\begin{subequations}\label{grp}
\begin{align}
a&=b+c\label{second}\\
d&=e+f+g\label{third}\\
h&=i+j\label{fourth}
\end{align}
\end{subequations}
By putting a \cn{label} command immediately after
\verb'\begin{subequations}' you can get a reference to the parent
  number; \verb'\eqref{grp}' from the above example would produce \eqref{grp}
  while \verb'\eqref{second}' would produce \eqref{second}.

An example at \url{https://tex.stackexchange.com/questions/220001/} shows
a variant of the above example, with numbering like (2.1), (2.1a),
\dots, rather than (2.1), (2.2a), \dots.  This is accomplished by using
\cn{tag} with a cross-reference to the principal component of the
subequation number.

\newpage

%%%%%%%%%%%%%%%%%%%%%%%%%%%%%%%%%%%%%%%%%%%%%%%%%%%%%%%%%%%%%%%%%%%%%%%%

\providecommand{\dotsref}{\leavevmode\unskip\ignorespaces}
\providecommand{\vertref}{\leavevmode\unskip\ignorespaces}

\section{Math symbols and math fonts}\label{mathsymbols}
\subsection{Classes of math symbols}

The symbols in a math formula fall into different classes that
correspond more or less to the part of speech each symbol would have if
the formula were expressed in words. Certain spacing and positioning
cues are traditionally used for the different symbol classes to increase
the readability of formulas.

\begin{center}
\begin{tabular}{clll}
\colhead{Class\\number}& \colhead{Mnemonic}& \colhead{Description\\(part
of speech)}& \colhead{Examples}\\\hline\noalign{\smallskip}
0& Ord& simple/ordinary (\qq{noun})& $A\;0\;\Phi\;\infty$\\
1& Op& prefix operator& $\sum\;\prod\;\int$\\
2& Bin& binary operator (conjunction)& ${+}\;{\cup}\;{\wedge}$\\
3& Rel& relation/comparison (verb)& ${=}\;{<}\;{\subset}$\\
4& Open& left/opening delimiter& $(\;{[}\;{\lbrace}\;{\langle}$\\
5& Close& right/closing delimiter& $)\;{]}\;{\rbrace}\;{\rangle}$\\
6& Punct& postfix/punctuation& ${.}\;{,}\;{;}\;{!}$\\
\end{tabular}
\end{center}
\begin{notes}
\item The distinction in \TeX{} between class 0 and an additional
class 7 has to do only with font selection issues, and it is immaterial
here.
\item Symbols of class 2 (Bin), notably the minus sign $-$, are
automatically printed by \LaTeX{} as class~0 (no space) if they do not
have a suitable left operand\mdash e.g., at the beginning of a math
formula or after an opening delimiter.
\end{notes}

The spacing for a few symbols follows tradition instead of the general
rule: although $/$ is (semantically speaking) of class~2, we write $k/2$
with no space around the slash rather than $k\mathbin{/}2$. And compare
\verb'p|q' $p\vert q$ (no space) with \verb'p\mid q' $p\mid q$
(class-3 spacing).

The proper way to define a new math symbol is discussed in
\emph{\LaTeXe{} font selection} \cite{fntguide}. It is not really
possible to give a useful synopsis here because one needs first to
understand the ramifications of font specifications. But supposing one
knows that a Cyrillic font named \fn{wncyr10} is available, here is a
minimal example showing how to define a \LaTeX{} command to print one
letter from that font as a math symbol:
\begin{verbatim}
% Declare that the combination of font attributes OT2/wncyr/m/n
% should select the wncyr font.
\DeclareFontShape{OT2}{wncyr}{m}{n}{<->wncyr10}{}
% Declare that the symbolic math font name "cyr" should resolve to
% OT2/wncyr/m/n.
\DeclareSymbolFont{cyr}{OT2}{wncyr}{m}{n}
% Declare that the command \Sh should print symbol 88 from the math font
% "cyr", and that the symbol class is 0 (= alphabetic = Ord).
\DeclareMathSymbol{\Sh}{\mathalpha}{cyr}{88}
\end{verbatim}

\subsection{Some symbols intentionally omitted here}

The following math symbols that are mentioned in the \LaTeX{} book
\cite{lamport} are intentionally omitted from this discussion because
they are superseded by equivalent symbols when the \pkg{amssymb} package
is loaded. If you are using the \pkg{amssymb} package anyway, the only
thing that you are likely to gain by using the alternate name is an
unnecessary increase in the number of fonts used by your document.
\begin{center}
\def\jdo#1{\cn{#1} \ $\csname #1\endcsname$}
\begin{tabular}{r@{\,, see \ }l}
\cn{Box}&\jdo{square}\\
\cn{Diamond}&\jdo{lozenge}\\
\cn{leadsto}&\jdo{rightsquigarrow}\\
\cn{Join}&\jdo{bowtie}\\
\cn{lhd}&\jdo{vartriangleleft}\\
\cn{unlhd}&\jdo{trianglelefteq}\\
\cn{rhd}&\jdo{vartriangleright}\\
\cn{unrhd}&\jdo{trianglerighteq}
\end{tabular}
\end{center}

Furthermore, there are \strong{many, many additional symbols} available
for \lat/ use above and beyond the ones included here. This list is not
intended to be comprehensive. For a much more comprehensive list of
symbols, including nonmathematically oriented ones, such as phonetic
alphabetic or dingbats, see \emph{The Comprehensive \LaTeX{} Symbol
List}~\cite{comprehensive}.  (Full font tables, ordered by font name,
for all the fonts covered by the comprehensive list are included in
the documentation provided by \TeX~Live: \texttt{texdoc rawtables}.
These tables do not include symbol names.)  Another source of symbol
information is the \pkg{unicode-math} package; see~\cite{uc-math}.

%%%%%%%%%%%%%%%%%%%%%%%%%%%%%%%%%%%%%%%%%%%%%%%%%%%%%%%%%%%%%%%%%%%%%%%%

%\newpage

\subsection{Alphabets and digits\nopunct}\label{alpha-digit}

\subsubsection{Latin letters and Arabic numerals}
The Latin letters are simple symbols, class~0. The default font for them
in math formulas is italic.
\begin{center}
\begin{tabular}{c}
  $A\,B\,C\,D\,E\,F\,G\,H\,I\,J\,K\,L\,M%
   \,N\,O\,P\,Q\,R\,S\,T\,U\,V\,W\,X\,Y\,Z$\\
  $a\,b\,c\,d\,e\,f\,g\,h\,i\,j\,k\,l\,m%
   \,n\,o\,p\,q\,r\,s\,t\,u\,v\,w\,x\,y\,z$
\end{tabular}
\end{center}
When adding an accent to an $i$ or $j$ in math, dotless variants can be
obtained with \cn{imath} and \cn{jmath}:
\begin{symlist}
\dosymbol{Var}{imath}{kernel}
\dosymbol{Var}{jmath}{kernel}
\symbox{\hat{\jmath}}{\string\hat\string{\string\jmath\string}}
\end{symlist}

Arabic numerals 0\ndash 9 are also of class~0. Their default font is
upright/roman.
\[0\,1\,2\,3\,4\,5\,6\,7\,8\,9\]

\subsubsection{Greek letters}
Like the Latin letters, the Greek letters are simple symbols, class~0.
For obscure historical reasons, the default font for lowercase Greek
letters in math formulas is italic while the default font for capital
Greek letters is upright/roman. (In other fields such as physics and
chemistry, however, the typographical traditions are somewhat
different.) The capital Greek letters not present in this list are the
letters that have the same appearance as some Latin letter: A for Alpha,
B for Beta, and so on. In the list of lowercase letters there is no
omicron because it would be identical in appearance to Latin $o$. In
practice, the Greek letters that have Latin look-alikes are seldom used
in math formulas, to avoid confusion.
\begin{symlist}[adjustheight=12pt]
\dosymbol{Var}{Gamma}{kernel}
\dosymbol{Var}{Delta}{kernel}
\dosymbol{Var}{Lambda}{kernel}
\dosymbol{Var}{Phi}{kernel}
\dosymbol{Var}{Pi}{kernel}
\dosymbol{Var}{Psi}{kernel}
\dosymbol{Var}{Sigma}{kernel}
\dosymbol{Var}{Theta}{kernel}
\dosymbol{Var}{Upsilon}{kernel}
\dosymbol{Var}{Xi}{kernel}
\dosymbol{Var}{Omega}{kernel}
\newcolumn
\dosymbol{Var}{alpha}{kernel}
\dosymbol{Var}{beta}{kernel}
\dosymbol{Var}{gamma}{kernel}
\dosymbol{Var}{delta}{kernel}
\dosymbol{Var}{epsilon}{kernel}
\dosymbol{Var}{zeta}{kernel}
\dosymbol{Var}{eta}{kernel}
\dosymbol{Var}{theta}{kernel}
\dosymbol{Var}{iota}{kernel}
\dosymbol{Var}{kappa}{kernel}
\dosymbol{Var}{lambda}{kernel}
\dosymbol{Var}{mu}{kernel}
\newcolumn
\dosymbol{Var}{nu}{kernel}
\dosymbol{Var}{xi}{kernel}
%\dosymbol{Var}{omicron}{??}
\dosymbol{Var}{pi}{kernel}
\dosymbol{Var}{rho}{kernel}
\dosymbol{Var}{sigma}{kernel}
\dosymbol{Var}{tau}{kernel}
\dosymbol{Var}{upsilon}{kernel}
\dosymbol{Var}{phi}{kernel}
\dosymbol{Var}{chi}{kernel}
\dosymbol{Var}{psi}{kernel}
\dosymbol{Var}{omega}{kernel}
\newcolumn
\dosymbol{Ord}{digamma}{amssymb}
\dosymbol{Var}{varepsilon}{kernel}
\dosymbol{Ord}{varkappa}{amssymb}
\dosymbol{Var}{varphi}{kernel}
\dosymbol{Var}{varpi}{kernel}
\dosymbol{Var}{varrho}{kernel}
\dosymbol{Var}{varsigma}{kernel}
\dosymbol{Var}{vartheta}{kernel}
\end{symlist}

\subsubsection{Other ``basic'' alphabetic symbols}
These are also class~0.
\begin{symlist}
\dosymbol{Ord}{aleph}{kernel}
\dosymbol{Ord}{beth}{amssymb}
\dosymbol{Ord}{daleth}{amssymb}
\dosymbol{Ord}{gimel}{amssymb}
\dosymbol{Ord}{complement}{amssymb}
\dosymbol{Ord}{ell}{kernel}
\dosymbol{Ord}{eth}{amssymb}
\dosymbol{Ord}{hbar}{amssymb}
\dosymbol{Ord}{hslash}{amssymb}
\dosymbol{Ord}{mho}{amssymb}
\dosymbol{Ord}{partial}{kernel}
\dosymbol{Ord}{wp}{kernel}
\dosymbol{Ord}{circledS}{amssymb}
\dosymbol{Ord}{Bbbk}{amssymb}
\dosymbol{Ord}{Finv}{amssymb}
\dosymbol{Ord}{Game}{amssymb}
\dosymbol{Ord}{Im}{kernel}
\dosymbol{Ord}{Re}{kernel}
\vfil
\end{symlist}
\begin{notes}
\end{notes}

\subsubsection{Math font switches}\label{mathfonts}
Not all of the fonts necessary to support comprehensive math font
switching are commonly available in a typical \LaTeX{} setup. Here are
the results of applying various font switches to a wide range of math
symbols when the standard set of Computer Modern fonts is in use. It can
be seen that the only symbols that respond correctly to all of the font
switches are the uppercase Latin letters. In fact, \emph{nearly all}
math symbols apart from Latin letters remain unaffected by font
switches; and although the lowercase Latin letters, capital Greek
letters, and numerals do respond properly to some font switches, they
produce bizarre results for other font switches. (Use of alternative
math font sets such as Lucida New Math may ameliorate the situation
somewhat.)
\[\renewcommand{\arraystretch}{1.3}
\begin{array}{cccccccc}
%\text{default}& \cn{mathbf}& \cn{mathsf}& \cn{mathit}& \cn{mathcal}&
%  \cn{mathscr}&  \cn{mathbb}& \cn{mathfrak}\\
\text{default}& \cn{mathbf}& \cn{mathrm}& \cn{mathsf}& \cn{mathit}&
  \cn{mathcal}& \cn{mathbb}& \cn{mathfrak}\\
\hline
\symrow{X} \\
\symrow{x} \\
\symrow{0} \\
\symrow{[\,]} \\
\symrow{+} \\
\symrow{-} \\
\symrow{=} \\
\symrow{\Xi} \\
\symrow{\xi} \\
\symrow{\infty} \\
\symrow{\aleph} \\
\symrow{\sum}\\
\symrow{\amalg} \\
\symrow{\Re} \\
\end{array}\]
A common desire is to get a bold version of a particular math symbol.
For those symbols where \cn{mathbf} is not applicable, the
\cn{boldsymbol} or \cn{pmb} commands can be used.
\begin{equation}
A_\infty + \pi A_0
\sim \mathbf{A}_{\boldsymbol{\infty}} \boldsymbol{+}
  \boldsymbol{\pi} \mathbf{A}_{\boldsymbol{0}}
\sim\pmb{A}_{\pmb{\infty}} \pmb{+}\pmb{\pi} \pmb{A}_{\pmb{0}}
\end{equation}
\begin{verbatim}
A_\infty + \pi A_0
\sim \mathbf{A}_{\boldsymbol{\infty}} \boldsymbol{+}
  \boldsymbol{\pi} \mathbf{A}_{\boldsymbol{0}}
\sim\pmb{A}_{\pmb{\infty}} \pmb{+}\pmb{\pi} \pmb{A}_{\pmb{0}}
\end{verbatim}
The \cn{boldsymbol} command is obtained preferably by using
the \pkg{bm} package, which provides a newer, more powerful version than
the one provided by the \pkg{amsmath} package. It is usually
ill-advised to apply \cn{boldsymbol} to more than one symbol at a time;
if such a need seems to arise, it more likely means that there is
another, better way of going about it.

\subsubsection{Blackboard Bold letters (\fn{msbm}; no lowercase)}
 Usage: \verb'\mathbb{R}'.  Requires \pkg{amsfonts}.
\[
\mathbb{A}\,\mathbb{B}\,\mathbb{C}\,\mathbb{D}\,\mathbb{E}\,\mathbb{F}
\,\mathbb{G}\,\mathbb{H}\,\mathbb{I}\,\mathbb{J}\,\mathbb{K}\,\mathbb{L}
\,\mathbb{M}\,\mathbb{N}\,\mathbb{O}\,\mathbb{P}\,\mathbb{Q}\,\mathbb{R}
\,\mathbb{S}\,\mathbb{T}\,\mathbb{U}\,\mathbb{V}\,\mathbb{W}\,\mathbb{X}
\,\mathbb{Y}\,\mathbb{Z}
\]
One lowercase letter is available with a distinct name:\qquad
$\Bbbk$\quad \cn{Bbbk}

%\newpage

\subsubsection{Calligraphic letters (\fn{cmsy}; no lowercase)} Usage:
\verb'\mathcal{M}'.
\[
\mathcal{A}\,\mathcal{B}\,\mathcal{C}\,\mathcal{D}\,\mathcal{E}
\,\mathcal{F}\,\mathcal{G}\,\mathcal{H}\,\mathcal{I}\,\mathcal{J}
\,\mathcal{K}\,\mathcal{L}\,\mathcal{M}\,\mathcal{N}\,\mathcal{O}
\,\mathcal{P}\,\mathcal{Q}\,\mathcal{R}\,\mathcal{S}\,\mathcal{T}
\,\mathcal{U}\,\mathcal{V}\,\mathcal{W}\,\mathcal{X}\,\mathcal{Y}
\,\mathcal{Z}
\]

\subsubsection{Non-CM calligraphic and script letters}
(\fn{rsfs}; no lowercase) Usage: \verb'\usepackage{mathrsfs}' \verb'\mathscr{B}'.
\[
\mathscr{A}\,\mathscr{B}\,\mathscr{C}\,\mathscr{D}\,\mathscr{E}
\,\mathscr{F}\,\mathscr{G}\,\mathscr{H}\,\mathscr{I}\,\mathscr{J}
\,\mathscr{K}\,\mathscr{L}\,\mathscr{M}\,\mathscr{N}\,\mathscr{O}
\,\mathscr{P}\,\mathscr{Q}\,\mathscr{R}\,\mathscr{S}\,\mathscr{T}
\,\mathscr{U}\,\mathscr{V}\,\mathscr{W}\,\mathscr{X}\,\mathscr{Y}
\,\mathscr{Z}
\]

\begingroup
\noindent
(\fn{eusm}; no lowercase) Usage: \verb'\usepackage{euscript}' \verb'\mathscr{E}'.
\renewcommand{\mathscr}{\EuScript}
\[
\mathscr{A}\,\mathscr{B}\,\mathscr{C}\,\mathscr{D}\,\mathscr{E}
\,\mathscr{F}\,\mathscr{G}\,\mathscr{H}\,\mathscr{I}\,\mathscr{J}
\,\mathscr{K}\,\mathscr{L}\,\mathscr{M}\,\mathscr{N}\,\mathscr{O}
\,\mathscr{P}\,\mathscr{Q}\,\mathscr{R}\,\mathscr{S}\,\mathscr{T}
\,\mathscr{U}\,\mathscr{V}\,\mathscr{W}\,\mathscr{X}\,\mathscr{Y}
\,\mathscr{Z}
\]
\endgroup

\subsubsection{Fraktur letters (\fn{eufm})}
Usage: \verb'\mathfrak{S}'.  Requires \pkg{amsfonts}.
\[
\mathfrak{A}\,\mathfrak{B}\,\mathfrak{C}\,\mathfrak{D}\,\mathfrak{E}
\,\mathfrak{F}\,\mathfrak{G}\,\mathfrak{H}\,\mathfrak{I}\,\mathfrak{J}
\,\mathfrak{K}\,\mathfrak{L}\,\mathfrak{M}\,\mathfrak{N}\,\mathfrak{O}
\,\mathfrak{P}\,\mathfrak{Q}\,\mathfrak{R}\,\mathfrak{S}\,\mathfrak{T}
\,\mathfrak{U}\,\mathfrak{V}\,\mathfrak{W}\,\mathfrak{X}\,\mathfrak{Y}
\,\mathfrak{Z}
\]
\[
\mathfrak{a}\,\mathfrak{b}\,\mathfrak{c}\,\mathfrak{d}\,\mathfrak{e}
\,\mathfrak{f}\,\mathfrak{g}\,\mathfrak{h}\,\mathfrak{i}\,\mathfrak{j}
\,\mathfrak{k}\,\mathfrak{l}\,\mathfrak{m}\,\mathfrak{n}\,\mathfrak{o}
\,\mathfrak{p}\,\mathfrak{q}\,\mathfrak{r}\,\mathfrak{s}\,\mathfrak{t}
\,\mathfrak{u}\,\mathfrak{v}\,\mathfrak{w}\,\mathfrak{x}\,\mathfrak{y}
\,\mathfrak{z}
\]

%%%%%%%%%%%%%%%%%%%%%%%%%%%%%%%%%%%%%%%%%%%%%%%%%%%%%%%%%%%%%%%%%%%%%%%%

\subsection{Miscellaneous simple symbols}
These symbols are also of class~0 (ordinary) which means
they do not have any built-in spacing.%
\begin{symlist}
\dosymbol{Ordx}{\#}{kernel}
\dosymbol{Ordx}{\&}{kernel}
\dosymbol{Ord}{angle}{amssymb}
\dosymbol{Ord}{backprime}{amssymb}
\dosymbol{Ord}{bigstar}{amssymb}
\dosymbol{Ord}{blacklozenge}{amssymb}
\dosymbol{Ord}{blacksquare}{amssymb}
\dosymbol{Ord}{blacktriangle}{amssymb}
\dosymbol{Ord}{blacktriangledown}{amssymb}
\dosymbol{Ord}{bot}{kernel}
\dosymbol{Ord}{clubsuit}{kernel}
\dosymbol{Ord}{diagdown}{amssymb}
\dosymbol{Ord}{diagup}{amssymb}
\dosymbol{Ord}{diamondsuit}{kernel}
\dosymbol{Ord}{emptyset}{kernel}
\dosymbol{Ord}{exists}{kernel}
\dosymbol{Ord}{flat}{kernel}
\dosymbol{Ord}{forall}{kernel}
\dosymbol{Ord}{heartsuit}{kernel}
\dosymbol{Ord}{infty}{kernel}
\dosymbol{Ord}{lozenge}{amssymb}
\dosymbol{Ord}{measuredangle}{amssymb}
\dosymbol{Ord}{nabla}{kernel}
\dosymbol{Ord}{natural}{kernel}
\dosymbol{Ord}{neg}{kernel}
\dosymbol{Ord}{nexists}{amssymb}
\dosymbol{Ord}{prime}{kernel}
\dosymbol{Ord}{sharp}{kernel}
\dosymbol{Ord}{spadesuit}{kernel}
\dosymbol{Ord}{sphericalangle}{amssymb}
\dosymbol{Ord}{square}{amssymb}
\dosymbol{Ord}{surd}{kernel}
\dosymbol{Ord}{top}{kernel}
\dosymbol{Ord}{triangle}{kernel}
\dosymbol{Ord}{triangledown}{amssymb}
\dosymbol{Ord}{varnothing}{amssymb}
\end{symlist}
\begin{notes}
\item A common mistake in the use of the symbols $\square$ and $\#$
  is to try to make them serve as binary operators or relation symbols
  without using a properly defined math symbol command. If you merely
  use the existing commands \cn{square} or \cn{\#} the intersymbol
  spacing will be incorrect because those commands produce a class-0
  symbol.
\item Synonyms: \alias{lnot}

\end{notes}

\subsection{Binary operator symbols\nopunct}
\begin{symlist}
\dosymbol{Binc}{*}{kernel}
\dosymbol{Binc}{+}{kernel}
\dosymbol{Binc}{-}{kernel}
\dosymbol{Bin}{amalg}{kernel}
\dosymbol{Bin}{ast}{kernel}
\dosymbol{Bin}{barwedge}{amssymb}
\dosymbol{Bin}{bigcirc}{kernel}
\dosymbol{Bin}{bigtriangledown}{kernel}
\dosymbol{Bin}{bigtriangleup}{kernel}
\dosymbol{Bin}{boxdot}{amssymb}
\dosymbol{Bin}{boxminus}{amssymb}
\dosymbol{Bin}{boxplus}{amssymb}
\dosymbol{Bin}{boxtimes}{amssymb}
\dosymbol{Bin}{bullet}{kernel}
\dosymbol{Bin}{cap}{kernel}
\dosymbol{Bin}{Cap}{amssymb}
\dosymbol{Bin}{cdot}{kernel}
\dosymbol{Bin}{centerdot}{amssymb}
\dosymbol{Bin}{circ}{kernel}
\dosymbol{Bin}{circledast}{amssymb}
\dosymbol{Bin}{circledcirc}{amssymb}
\dosymbol{Bin}{circleddash}{amssymb}
\dosymbol{Bin}{cup}{kernel}
\dosymbol{Bin}{Cup}{amssymb}
\dosymbol{Bin}{curlyvee}{amssymb}
\dosymbol{Bin}{curlywedge}{amssymb}
\dosymbol{Bin}{dagger}{kernel}
\dosymbol{Bin}{ddagger}{kernel}
\dosymbol{Bin}{diamond}{kernel}
\dosymbol{Bin}{div}{kernel}
\dosymbol{Bin}{divideontimes}{amssymb}
\dosymbol{Bin}{dotplus}{amssymb}
\dosymbol{Bin}{doublebarwedge}{amssymb}
\dosymbol{Bin}{gtrdot}{amssymb}
\dosymbol{Bin}{intercal}{amssymb}
\dosymbol{Bin}{leftthreetimes}{amssymb}
\dosymbol{Bin}{lessdot}{amssymb}
\dosymbol{Bin}{ltimes}{amssymb}
\dosymbol{Bin}{mp}{kernel}
\dosymbol{Bin}{odot}{kernel}
\dosymbol{Bin}{ominus}{kernel}
\dosymbol{Bin}{oplus}{kernel}
\dosymbol{Bin}{oslash}{kernel}
\dosymbol{Bin}{otimes}{kernel}
\dosymbol{Bin}{pm}{kernel}
\dosymbol{Bin}{rightthreetimes}{amssymb}
\dosymbol{Bin}{rtimes}{amssymb}
\dosymbol{Bin}{setminus}{kernel}
\dosymbol{Bin}{smallsetminus}{amssymb}
\dosymbol{Bin}{sqcap}{kernel}
\dosymbol{Bin}{sqcup}{kernel}
\dosymbol{Bin}{star}{kernel}
\dosymbol{Bin}{times}{kernel}
\dosymbol{Bin}{triangleleft}{kernel}
\dosymbol{Bin}{triangleright}{kernel}
\dosymbol{Bin}{uplus}{kernel}
\dosymbol{Bin}{vee}{kernel}
\dosymbol{Bin}{veebar}{amssymb}
\dosymbol{Bin}{wedge}{kernel}
\dosymbol{Bin}{wr}{kernel}
\end{symlist}
\begin{notes}
\synonyms \alias{land}, \alias{lor}, \alias{doublecup}, \alias{doublecap}
\end{notes}

\subsection{Relation symbols:
    \texorpdfstring{$<$ $=$ $>$ $\succ$ $\sim$}{< + > succeed ~}
    and variants\nopunct}
\begin{symlist}[adjustheight=10pt]
\dosymbol{Relc}{<}{kernel}
\dosymbol{Relc}{=}{kernel}
\dosymbol{Relc}{>}{kernel}
\dosymbol{Rel}{approx}{kernel}
\dosymbol{Rel}{approxeq}{amssymb}
\dosymbol{Rel}{asymp}{kernel}
\dosymbol{Rel}{backsim}{amssymb}
\dosymbol{Rel}{backsimeq}{amssymb}
\dosymbol{Rel}{bumpeq}{amssymb}
\dosymbol{Rel}{Bumpeq}{amssymb}
\dosymbol{Rel}{circeq}{amssymb}
\dosymbol{Rel}{cong}{kernel}
\dosymbol{Rel}{curlyeqprec}{amssymb}
\dosymbol{Rel}{curlyeqsucc}{amssymb}
\dosymbol{Rel}{doteq}{kernel}
\dosymbol{Rel}{doteqdot}{amssymb}
\dosymbol{Rel}{eqcirc}{amssymb}
\dosymbol{Rel}{eqsim}{amssymb}
\dosymbol{Rel}{eqslantgtr}{amssymb}
\dosymbol{Rel}{eqslantless}{amssymb}
\dosymbol{Rel}{equiv}{kernel}
\dosymbol{Rel}{fallingdotseq}{amssymb}
\dosymbol{Rel}{geq}{kernel}
\dosymbol{Rel}{geqq}{amssymb}
\dosymbol{Rel}{geqslant}{amssymb}
\dosymbol{Rel}{gg}{kernel}
\dosymbol{Rel}{ggg}{amssymb}
\dosymbol{Rel}{gnapprox}{amssymb}
\dosymbol{Rel}{gneq}{amssymb}
\dosymbol{Rel}{gneqq}{amssymb}
\dosymbol{Rel}{gnsim}{amssymb}
\dosymbol{Rel}{gtrapprox}{amssymb}
\dosymbol{Rel}{gtreqless}{amssymb}
\dosymbol{Rel}{gtreqqless}{amssymb}
\dosymbol{Rel}{gtrless}{amssymb}
\dosymbol{Rel}{gtrsim}{amssymb}
\dosymbol{Rel}{gvertneqq}{amssymb}
\dosymbol{Rel}{leq}{kernel}
\dosymbol{Rel}{leqq}{amssymb}
\dosymbol{Rel}{leqslant}{amssymb}
\dosymbol{Rel}{lessapprox}{amssymb}
\dosymbol{Rel}{lesseqgtr}{amssymb}
\dosymbol{Rel}{lesseqqgtr}{amssymb}
\dosymbol{Rel}{lessgtr}{amssymb}
\dosymbol{Rel}{lesssim}{amssymb}
\dosymbol{Rel}{ll}{kernel}
\dosymbol{Rel}{lll}{amssymb}
\dosymbol{Rel}{lnapprox}{amssymb}
\dosymbol{Rel}{lneq}{amssymb}
\dosymbol{Rel}{lneqq}{amssymb}
\dosymbol{Rel}{lnsim}{amssymb}
\dosymbol{Rel}{lvertneqq}{amssymb}
\dosymbol{Rel}{ncong}{amssymb}
\dosymbol{Rel}{neq}{kernel}
\dosymbol{Rel}{ngeq}{amssymb}
\dosymbol{Rel}{ngeqq}{amssymb}
\dosymbol{Rel}{ngeqslant}{amssymb}
\dosymbol{Rel}{ngtr}{amssymb}
\dosymbol{Rel}{nleq}{amssymb}
\dosymbol{Rel}{nleqq}{amssymb}
\dosymbol{Rel}{nleqslant}{amssymb}
\dosymbol{Rel}{nless}{amssymb}
\dosymbol{Rel}{nprec}{amssymb}
\dosymbol{Rel}{npreceq}{amssymb}
\dosymbol{Rel}{nsim}{amssymb}
\dosymbol{Rel}{nsucc}{amssymb}
\dosymbol{Rel}{nsucceq}{amssymb}
\dosymbol{Rel}{prec}{kernel}
\dosymbol{Rel}{precapprox}{amssymb}
\dosymbol{Rel}{preccurlyeq}{amssymb}
\dosymbol{Rel}{preceq}{kernel}
\dosymbol{Rel}{precnapprox}{amssymb}
\dosymbol{Rel}{precneqq}{amssymb}
\dosymbol{Rel}{precnsim}{amssymb}
\dosymbol{Rel}{precsim}{amssymb}
\dosymbol{Rel}{risingdotseq}{amssymb}
\dosymbol{Rel}{sim}{kernel}
\dosymbol{Rel}{simeq}{kernel}
\dosymbol{Rel}{succ}{kernel}
\dosymbol{Rel}{succapprox}{amssymb}
\dosymbol{Rel}{succcurlyeq}{amssymb}
\dosymbol{Rel}{succeq}{kernel}
\dosymbol{Rel}{succnapprox}{amssymb}
\dosymbol{Rel}{succneqq}{amssymb}
\dosymbol{Rel}{succnsim}{amssymb}
\dosymbol{Rel}{succsim}{amssymb}
\dosymbol{Rel}{thickapprox}{amssymb}
\dosymbol{Rel}{thicksim}{amssymb}
\dosymbol{Rel}{triangleq}{amssymb}
\end{symlist}
\begin{notes}
  \synonyms \alias{ne}, \alias{le}, \alias{ge}, \alias{Doteq}, \alias{llless}, \alias{gggtr}
\end{notes}

\subsection{Relation symbols: arrows}
See also \secref{notations}.
\begin{symlist}[adjustheight=10pt]
\dosymbol{Rel}{circlearrowleft}{amssymb}
\dosymbol{Rel}{circlearrowright}{amssymb}
\dosymbol{Rel}{curvearrowleft}{amssymb}
\dosymbol{Rel}{curvearrowright}{amssymb}
\dosymbol{Rel}{downdownarrows}{amssymb}
\dosymbol{Rel}{downharpoonleft}{amssymb}
\dosymbol{Rel}{downharpoonright}{amssymb}
\dosymbol{Rel}{hookleftarrow}{kernel}
\dosymbol{Rel}{hookrightarrow}{kernel}
\dosymbol{Rel}{leftarrow}{kernel}
\dosymbol{Rel}{Leftarrow}{kernel}
\dosymbol{Rel}{leftarrowtail}{amssymb}
\dosymbol{Rel}{leftharpoondown}{kernel}
\dosymbol{Rel}{leftharpoonup}{kernel}
\dosymbol{Rel}{leftleftarrows}{amssymb}
\dosymbol{Rel}{leftrightarrow}{kernel}
\dosymbol{Rel}{Leftrightarrow}{kernel}
\dosymbol{Rel}{leftrightarrows}{amssymb}
\dosymbol{Rel}{leftrightharpoons}{amssymb}
\dosymbol{Rel}{leftrightsquigarrow}{amssymb}
\dosymbol{Rel}{Lleftarrow}{amssymb}
\dosymbol{Rel}{longleftarrow}{kernel}
\dosymbol{Rel}{Longleftarrow}{kernel}
\dosymbol{Rel}{longleftrightarrow}{kernel}
\dosymbol{Rel}{Longleftrightarrow}{kernel}
\dosymbol{Rel}{longmapsto}{kernel}
\dosymbol{Rel}{longrightarrow}{kernel}
\dosymbol{Rel}{Longrightarrow}{kernel}
\dosymbol{Rel}{looparrowleft}{amssymb}
\dosymbol{Rel}{looparrowright}{amssymb}
\dosymbol{Rel}{Lsh}{amssymb}
\dosymbol{Rel}{mapsto}{kernel}
\dosymbol{Rel}{multimap}{amssymb}
\dosymbol{Rel}{nLeftarrow}{amssymb}
\dosymbol{Rel}{nLeftrightarrow}{amssymb}
\dosymbol{Rel}{nRightarrow}{amssymb}
\dosymbol{Rel}{nearrow}{kernel}
\dosymbol{Rel}{nleftarrow}{amssymb}
\dosymbol{Rel}{nleftrightarrow}{amssymb}
\dosymbol{Rel}{nrightarrow}{amssymb}
\dosymbol{Rel}{nwarrow}{kernel}
\dosymbol{Rel}{rightarrow}{kernel}
\dosymbol{Rel}{Rightarrow}{kernel}
\dosymbol{Rel}{rightarrowtail}{amssymb}
\dosymbol{Rel}{rightharpoondown}{kernel}
\dosymbol{Rel}{rightharpoonup}{kernel}
\dosymbol{Rel}{rightleftarrows}{amssymb}
\dosymbol{Rel}{rightleftharpoons}{amssymb}
\dosymbol{Rel}{rightrightarrows}{amssymb}
\dosymbol{Rel}{rightsquigarrow}{amssymb}
\dosymbol{Rel}{Rrightarrow}{amssymb}
\dosymbol{Rel}{Rsh}{amssymb}
\dosymbol{Rel}{searrow}{kernel}
\dosymbol{Rel}{swarrow}{kernel}
\dosymbol{Rel}{twoheadleftarrow}{amssymb}
\dosymbol{Rel}{twoheadrightarrow}{amssymb}
\dosymbol{Rel}{upharpoonleft}{amssymb}
\dosymbol{Rel}{upharpoonright}{amssymb}
\dosymbol{Rel}{upuparrows}{amssymb}
\end{symlist}
\begin{notes}
  \synonyms \alias{gets}, \alias{to}, \alias{restriction}
\end{notes}

\subsection{Relation symbols: miscellaneous\nopunct}
\begin{symlist}[adjustheight=10pt]
\dosymbol{Rel}{backepsilon}{amssymb}
\dosymbol{Rel}{because}{amssymb}
\dosymbol{Rel}{between}{amssymb}
\dosymbol{Rel}{blacktriangleleft}{amssymb}
\dosymbol{Rel}{blacktriangleright}{amssymb}
\dosymbol{Rel}{bowtie}{kernel}
\dosymbol{Rel}{dashv}{kernel}
\dosymbol{Rel}{frown}{kernel}
\dosymbol{Rel}{in}{kernel}
\dosymbol{Rel}{mid}{kernel}
\dosymbol{Rel}{models}{kernel}
\dosymbol{Rel}{ni}{kernel}
\dosymbol{Rel}{nmid}{amssymb}
\dosymbol{Rel}{notin}{kernel}
\dosymbol{Rel}{nparallel}{amssymb}
\dosymbol{Rel}{nshortmid}{amssymb}
\dosymbol{Rel}{nshortparallel}{amssymb}
\dosymbol{Rel}{nsubseteq}{amssymb}
\dosymbol{Rel}{nsubseteqq}{amssymb}
\dosymbol{Rel}{nsupseteq}{amssymb}
\dosymbol{Rel}{nsupseteqq}{amssymb}
\dosymbol{Rel}{ntriangleleft}{amssymb}
\dosymbol{Rel}{ntrianglelefteq}{amssymb}
\dosymbol{Rel}{ntriangleright}{amssymb}
\dosymbol{Rel}{ntrianglerighteq}{amssymb}
\dosymbol{Rel}{nvdash}{amssymb}
\dosymbol{Rel}{nVdash}{amssymb}
\dosymbol{Rel}{nvDash}{amssymb}
\dosymbol{Rel}{nVDash}{amssymb}
\dosymbol{Rel}{parallel}{kernel}
\dosymbol{Rel}{perp}{kernel}
\dosymbol{Rel}{pitchfork}{amssymb}
\dosymbol{Rel}{propto}{kernel}
\dosymbol{Rel}{shortmid}{amssymb}
\dosymbol{Rel}{shortparallel}{amssymb}
\dosymbol{Rel}{smallfrown}{amssymb}
\dosymbol{Rel}{smallsmile}{amssymb}
\dosymbol{Rel}{smile}{kernel}
\dosymbol{Rel}{sqsubset}{amssymb}
\dosymbol{Rel}{sqsubseteq}{kernel}
\dosymbol{Rel}{sqsupset}{amssymb}
\dosymbol{Rel}{sqsupseteq}{kernel}
\dosymbol{Rel}{subset}{kernel}
\dosymbol{Rel}{Subset}{amssymb}
\dosymbol{Rel}{subseteq}{kernel}
\dosymbol{Rel}{subseteqq}{amssymb}
\dosymbol{Rel}{subsetneq}{amssymb}
\dosymbol{Rel}{subsetneqq}{amssymb}
\dosymbol{Rel}{supset}{kernel}
\dosymbol{Rel}{Supset}{amssymb}
\dosymbol{Rel}{supseteq}{kernel}
\dosymbol{Rel}{supseteqq}{amssymb}
\dosymbol{Rel}{supsetneq}{amssymb}
\dosymbol{Rel}{supsetneqq}{amssymb}
\dosymbol{Rel}{therefore}{amssymb}
\dosymbol{Rel}{trianglelefteq}{amssymb}
\dosymbol{Rel}{trianglerighteq}{amssymb}
\dosymbol{Rel}{varpropto}{amssymb}
\dosymbol{Rel}{varsubsetneq}{amssymb}
\dosymbol{Rel}{varsubsetneqq}{amssymb}
\dosymbol{Rel}{varsupsetneq}{amssymb}
\dosymbol{Rel}{varsupsetneqq}{amssymb}
\dosymbol{Rel}{vartriangle}{amssymb}
\dosymbol{Rel}{vartriangleleft}{amssymb}
\dosymbol{Rel}{vartriangleright}{amssymb}
\dosymbol{Rel}{vdash}{kernel}
\dosymbol{Rel}{Vdash}{amssymb}
\dosymbol{Rel}{vDash}{amssymb}
\dosymbol{Rel}{Vvdash}{amssymb}
\end{symlist}
\begin{notes}
  \synonyms \alias{owns}
\end{notes}

\subsection{Cumulative (variable-size) operators\nopunct}
\begin{symlist}[adjustcols=-1]
\openup3pt
\dosymbol{COi}{int}{kernel}
\dosymbol{COi}{oint}{kernel}
\dosymbol{COs}{bigcap}{kernel}
\dosymbol{COs}{bigcup}{kernel}
\dosymbol{COs}{bigodot}{kernel}
\dosymbol{COs}{bigoplus}{kernel}
\dosymbol{COs}{bigotimes}{kernel}
\dosymbol{COs}{bigsqcup}{kernel}
\dosymbol{COs}{biguplus}{kernel}
\dosymbol{COs}{bigvee}{kernel}
\dosymbol{COs}{bigwedge}{kernel}
\dosymbol{COs}{coprod}{kernel}
\dosymbol{COs}{prod}{kernel}
\dosymbol{COs}{smallint}{kernel}
\dosymbol{COs}{sum}{kernel}
\end{symlist}

\subsection{Punctuation\nopunct}
\begin{symlist}[adjustcols=-4]
\openup2pt
\dosymbol{Ordc}{.}{kernel}
\dosymbol{Ordc}{/}{kernel}
\dosymbol{Ordc}{|}{kernel}
\dosymbol{Punc}{,}{kernel}
\dosymbol{Punc}{;}{kernel}
\dosymbol{Pun}{colon}{kernel}
\dosymbol{Relc}{:}{kernel}
%\dosymbol{DeR}{!}{kernel}
%\dosymbol{DeR}{?}{kernel}
\dosymbol{Punc}{!}{kernel}
\dosymbol{Punc}{?}{kernel}
\dosymbol{Inn}{dotsb}{kernel}
\dosymbol{Inn}{dotsc}{kernel}
\dosymbol{Inn}{dotsi}{kernel}
\dosymbol{Inn}{dotsm}{kernel}
\dosymbol{Inn}{dotso}{kernel}
\dosymbol{Inn}{ddots}{kernel}
\dosymbol{Ord}{vdots}{kernel}
\end{symlist}
\begin{notes}
\item The \verb':' by itself produces a colon with
  class-3 (relation) spacing. The command \cn{colon} produces special
  spacing for use in constructions such as \verb'f\colon A\to B'
  $f\colon A\to B$.
\item Although the commands \cn{cdots} and \cn{ldots} are frequently
  used, we recommend the more semantically oriented commands
  \cn{dotsb} \cn{dotsc} \cn{dotsi} \cn{dotsm} \cn{dotso} for most
  purposes\dotsref.
\end{notes}

\subsection{Pairing delimiters (extensible)}\label{pair-delims}
See Section~\ref{delim} for more information.
\begin{symlist}
\openup7pt
\dosymbol{DeLRc}{(}{)}{kernel}
\dosymbol{DeLRc}{[}{]}{kernel}
\dosymbol{DeLR}{lbrace}{rbrace}{kernel}
\dosymbol{DeLR}{lvert}{rvert}{kernel}
\dosymbol{DeLR}{lVert}{rVert}{kernel}
\dosymbol{DeLR}{langle}{rangle}{kernel}
\dosymbol{DeLR}{lceil}{rceil}{kernel}
\dosymbol{DeLR}{lfloor}{rfloor}{kernel}
\dosymbol{DeLR}{lgroup}{rgroup}{kernel}
\dosymbol{DeLR}{lmoustache}{rmoustache}{kernel}
\end{symlist}

\subsection{Nonpairing extensible symbols\nopunct}
\begin{symlist}
\dosymbol{DeB}{vert}{kernel}
\dosymbol{DeB}{Vert}{kernel}
\dosymbol{DeBc}{/}{kernel}
\dosymbol{DeB}{backslash}{kernel}
\dosymbol{DeB}{arrowvert}{kernel}
\dosymbol{DeB}{Arrowvert}{kernel}
\dosymbol{DeB}{bracevert}{kernel}
\end{symlist}
\begin{notes}
\item Using \cn{vert}, \verb'|', \cn{Vert}, or \cn{|} for paired
delimiters is not recommended\vertref.  Instead, use delimiters from
the list in Section~\ref{pair-delims}.
  \synonyms \alias{|}
\end{notes}

\subsection{Extensible vertical arrows\nopunct}
\begin{symlist}
\dosymbol{DeA}{uparrow}{kernel}
\dosymbol{DeA}{Uparrow}{kernel}
\dosymbol{DeA}{downarrow}{kernel}
\dosymbol{DeA}{Downarrow}{kernel}
\dosymbol{DeA}{updownarrow}{kernel}
\dosymbol{DeA}{Updownarrow}{kernel}
\end{symlist}

\subsection{Math accents\nopunct}\label{accents}
\begin{symlist}[adjustcols=1]
\dosymbol{Acc}{acute}{kernel}
\dosymbol{Acc}{grave}{kernel}
\dosymbol{Acc}{ddot}{kernel}
\dosymbol{Acc}{tilde}{kernel}
\dosymbol{Acc}{bar}{kernel}
\dosymbol{Acc}{breve}{kernel}
\dosymbol{Acc}{check}{kernel}
\dosymbol{Acc}{hat}{kernel}
\dosymbol{Acc}{vec}{kernel}
\dosymbol{Acc}{dot}{kernel}
\dosymbol{Acc}{ddot}{amsmath}
\dosymbol{Acc}{dddot}{amsmath}
\dosymbol{Acc}{mathring}{amsmath}
\dosymbol{Accw}{widetilde}{kernel}
\dosymbol{Accw}{widehat}{kernel}
\end{symlist}

\subsection{Named operators}
These operators are represented by a multiletter abbreviation.
\shiftlistright40pt
\begin{symlist}[adjustcols=-1]
\dosymbol{Opn}{arccos}{kernel}
\dosymbol{Opn}{arcsin}{kernel}
\dosymbol{Opn}{arctan}{kernel}
\dosymbol{Opn}{arg}{kernel}
\dosymbol{Opn}{cos}{kernel}
\dosymbol{Opn}{cosh}{kernel}
\dosymbol{Opn}{cot}{kernel}
\dosymbol{Opn}{coth}{kernel}
\dosymbol{Opn}{csc}{kernel}
\dosymbol{Opn}{deg}{kernel}
\dosymbol{Opn}{det}{kernel}
\dosymbol{Opn}{dim}{kernel}
\dosymbol{Opn}{exp}{kernel}
\dosymbol{Opn}{gcd}{kernel}
\dosymbol{Opn}{hom}{kernel}
\dosymbol{Opn}{inf}{kernel}
\dosymbol{Opn}{injlim}{kernel}
\dosymbol{Opn}{ker}{kernel}
\dosymbol{Opn}{lg}{kernel}
\dosymbol{Opn}{lim}{kernel}
\dosymbol{Opn}{liminf}{kernel}
\dosymbol{Opn}{limsup}{kernel}
\dosymbol{Opn}{ln}{kernel}
\dosymbol{Opn}{log}{kernel}
\dosymbol{Opn}{max}{kernel}
\dosymbol{Opn}{min}{kernel}
\dosymbol{Opn}{Pr}{kernel}
\dosymbol{Opn}{projlim}{kernel}
\dosymbol{Opn}{sec}{kernel}
\dosymbol{Opn}{sin}{kernel}
\dosymbol{Opn}{sinh}{kernel}
\dosymbol{Opn}{sup}{kernel}
\dosymbol{Opn}{tan}{kernel}
\dosymbol{Opn}{tanh}{kernel}
\dosymbol{Opn}{varinjlim}{kernel}
\dosymbol{Opn}{varprojlim}{kernel}
\dosymbol{Opn}{varliminf}{kernel}
\dosymbol{Opn}{varlimsup}{kernel}
  \end{symlist}

To define additional named operators outside the above list, use the
\cn{DeclareMathOperator} command; for example, after
\begin{verbatim}
\DeclareMathOperator{\rank}{rank}
\DeclareMathOperator{\esssup}{ess\,sup}
\end{verbatim}
one could write
\begin{center}
\begin{tabular}{rl}
\verb'\rank(x)'& $\rank(x)$\\
\verb'\esssup(y,z)'& $\esssup(y,z)$
\end{tabular}
\end{center}
The star form \cn{DeclareMathOperator*} creates an operator that takes
limits in a displayed formula, such as $\sup$ or $\max$.

When predefining such a named operator is problematic (e.g., when using
one in the title or abstract of an article), there is an alternative
form that can be used directly:
\[\verb'\operatorname{rank}(x)'\quad
  \rightarrow\quad\operatorname{rank}(x)\]

%%%%%%%%%%%%%%%%%%%%%%%%%%%%%%%%%%%%%%%%%%%%%%%%%%%%%%%%%%%%%%%%%%%%%%%%

\section{Notations}
\label{notations}

\subsection{Top and bottom embellishments}

These are visually similar to accents but generally span multiple
symbols rather than being applied to a single base symbol. For ease of
reference, \cn{widetilde} and \cn{widehat} are redundantly included here
and in the table of math accents.
\begin{symlist}
\dosymbol{Accw}{widetilde}{kernel}
\dosymbol{Accw}{widehat}{kernel}
\dosymbol{Accw}{overline}{kernel}
\dosymbol{Accw}{underline}{kernel}
\dosymbol{Accw}{overbrace}{kernel}
\dosymbol{Accw}{underbrace}{kernel}
\dosymbol{Accw}{overleftarrow}{kernel}
\dosymbol{Accw}{underleftarrow}{amsmath}
\dosymbol{Accw}{overrightarrow}{kernel}
\dosymbol{Accw}{underrightarrow}{amsmath}
\dosymbol{Accw}{overleftrightarrow}{amsmath}
\dosymbol{Accw}{underleftrightarrow}{amsmath}
\end{symlist}

\subsection{Extensible arrows}

\cn{xleftarrow} and \cn{xrightarrow} produce
arrows\index{arrows!extensible} that extend automatically to accommodate
unusually wide subscripts or superscripts. These commands take one
optional argument (the subscript) and one mandatory argument (the
superscript, possibly empty):
\begin{equation}
A\xleftarrow{n+\mu-1}B \xrightarrow[T]{n\pm i-1}C
\end{equation}
\begin{verbatim}
  \xleftarrow{n+\mu-1}\quad \xrightarrow[T]{n\pm i-1}
\end{verbatim}

\subsection{Affixing symbols to other symbols}

In addition to the standard accents (Section~\ref{accents}), other
symbols can be placed above or below a base symbol with the \cn{overset}
and \cn{underset} commands. For example, writing \verb|\overset{*}{X}|
will place a superscript-size $*$ above the $X$, thus: $\overset{*}{X}$.
See also the description of \cn{sideset} in \secref{sideset}.

\subsection{Matrices}\label{ss:matrix}

The environments \env{pmatrix}, \env{bmatrix}, \env{Bmatrix},
\env{vmatrix}, and \env{Vmatrix} have (respectively) $(\,)$, $[\,]$,
$\lbrace\,\rbrace$, $\lvert\,\rvert$, and $\lVert\,\rVert$ delimiters
built in. There is also a \env{matrix} environment without delimiters
and an \env{array} environment that can be used to obtain left alignment
or other variations in the column specs.
\begin{center}
\begin{minipage}{.4\columnwidth}
\begin{verbatim}
\begin{pmatrix}
\alpha& \beta^{*}\\
\gamma^{*}& \delta
\end{pmatrix}
\end{verbatim}
\end{minipage}
\qquad
\begin{minipage}{.4\columnwidth}
\[
\begin{pmatrix}
\alpha& \beta^{*}\\
\gamma^{*}& \delta
\end{pmatrix}
\]
\end{minipage}
\end{center}
To produce a small matrix suitable for use in text, there is a
\env{smallmatrix} environment (e.g.,
\begin{math}
\bigl( \begin{smallmatrix}
  a&b\\ c&d
\end{smallmatrix} \bigr)
\end{math})
that comes closer to fitting within a single text line than a normal
matrix. This example was produced by
\begin{verbatim}
\bigl( \begin{smallmatrix}
  a&b\\ c&d
\end{smallmatrix} \bigr)
\end{verbatim}
By default, all elements in a matrix are centered horizontally.
The \pkg{mathtools} package provides starred versions of all the matrix
environments that facilitate other alignments.  That package also provides
fenced versions of \env{smallmatrix} with parallel names in both starred
and nonstarred versions.

To produce a row of dots in a matrix\index{matrices!ellipsis
  dots}\index{ellipsis dots!in matrices}\index{dots|see{ellipsis dots}}
spanning a given number of columns, use \cn{hdotsfor}. For example,
\verb'\hdotsfor{3}' in the second column of a four-column matrix will
print a row of dots across the final three columns.

For piecewise function definitions there is a \env{cases} environment:
\begin{verbatim}
P_{r-j}=\begin{cases}
    0&  \text{if $r-j$ is odd},\\
    r!\,(-1)^{(r-j)/2}&  \text{if $r-j$ is even}.
  \end{cases}
\end{verbatim}
Notice the use of \cn{text} and the embedded math.

\begin{notes}
  \singlenote The plain \TeX{} form \verb'\matrix{...\cr...\cr}' and the
  related commands \cn{pmatrix}, \cn{cases} should be avoided in
  \LaTeX{} (and when the \pkg{amsmath} package is loaded they are
  disabled).
\end{notes}

\subsection{Math spacing commands}

When the \pkg{amsmath} package is used, all of these math spacing
commands can be used both in and out of math mode.
\begin{center}\begin{tabular}{llllll}
Abbrev.& Spelled out& Example & Abbrev.& Spelled out& Example\\
\hline
\strut & no space& \spx{}& & no space& \spx{}\\
\cn{\,}& \cn{thinspace}& \spx{\,}&
  \cn{!}& \cn{negthinspace}& \spx{\!}\\
\cn{\:}& \cn{medspace}& \spx{\:}&
  & \cn{negmedspace}& \spx{\negmedspace}\\
\cn{\;}& \cn{thickspace}& \spx{\;}&
  & \cn{negthickspace}& \spx{\negthickspace}\\
& \cn{quad}& \spx{\quad}\\
& \cn{qquad}& \spx{\qquad}
\end{tabular}\end{center}
For finer control over math spacing, use \cn{mspace}
and `math units'. One math unit, or \verb|mu|, is equal to 1/18 em. Thus to
get a negative half \cn{quad} write \verb|\mspace{-9.0mu}|.

There are also three commands that leave a space equal to the height
and/or width of a given fragment of \lat/ material:
\begin{center}\begin{tabular}{ll}
\colhead{Example}& \colhead{Result}\\
\hline
\verb'\phantom{XXX}'& space as wide and high as three X's\strut \\
\verb'\hphantom{XXX}'& space as wide as three X's; height 0\\
\verb'\vphantom{X}'& space of width 0, height = height of X
\end{tabular}\end{center}

\subsection{Dots}\label{dots}

For preferred placement of ellipsis dots (raised or on-line) in various
contexts there is no general consensus. It may therefore be considered a
matter of taste.  In most situations, the generic \cn{dots} can be used,
and \pkg{amsmath} will interpret it in the manner preferred by the AMS,
namely low dots (\cn{ldots}) between commas or raised dots (\cn{cdots})
between binary operators and relations, etc.  If what follows the dots is
ambiguous as to the choice, the specific form of the command can be used.
However, by using the semantically oriented commands
\begin{itemize}
\setlength{\itemsep}{0pt}
\item \cn{dotsc} for \qq{dots with commas}
\item \cn{dotsb} for \qq{dots with binary operators/relations}
\item \cn{dotsm} for \qq{multiplication dots}
\item \cn{dotsi} for \qq{dots with integrals}
\item \cn{dotso} for \qq{other dots} (none of the above)
\end{itemize}
instead of \cn{ldots} and \cn{cdots}, you make it possible for your
document to be adapted to different conventions on the fly, in case (for
example) you have to submit it to a publisher who insists on following
house tradition in this respect. The default treatment for the various
kinds follows American Mathematical Society conventions:
\begin{center}
\vspace{-\topsep}
\begin{tabular}{@{}ll@{}}
\begin{minipage}[t]{.50\textwidth}
\small
\begin{verbatim}
We have the series $A_1,A_2,\dotsc$,
the regional sum $A_1+A_2+\dotsb$,
the orthogonal product $A_1A_2\dotsm$,
and the infinite integral
\[\int_{A_1}\int_{A_2}\dotsi\].
\end{verbatim}
\end{minipage}
&
\begin{minipage}[t]{.48\textwidth}
\noindent
We have the series $A_1,A_2,\dotsc$,
the regional sum $A_1+A_2+\dotsb$,
the orthogonal product $A_1A_2\dotsm$,
and the infinite integral
\[\int_{A_1}\int_{A_2}\dotsi.\]
\end{minipage}
\end{tabular}
\end{center}

\subsection{Nonbreaking dashes}

The command \cn{nobreakdash} suppresses the possibility
of a linebreak after the following hyphen or dash. For example, if you
write `pages 1\ndash 9' as \verb|pages 1\nobreakdash--9| then a linebreak will
never occur between the dash and the 9. You can also use
\cn{nobreakdash} to prevent undesirable hyphenations in combinations
like \verb|$p$-adic|. For frequent use, it's advisable to make abbreviations,
e.g.,
\begin{verbatim}
\newcommand{\p}{$p$\nobreakdash}% for "\p adic" ("p-adic")
\newcommand{\Ndash}{\nobreakdash\textendash}% for "pages 1\Ndash 9"
%    For "\n dimensional" ("n-dimensional"):
\newcommand{\n}{$n$\nobreakdash-\hspace{0pt}}
\end{verbatim}
The last example shows how to prohibit a linebreak after the hyphen but
allow normal hyphenation in the following word. (Add a zero-width space
after the hyphen.)

\subsection{Roots}
The command \cn{sqrt} produces a square root. To specify an explicit
radix, give it as an optional argument.
\[
\verb'\sqrt{\frac{n}{n-1} S}'\quad\sqrt{\frac{n}{n-1} S}, \qquad
\verb'\sqrt[3]{2}'\quad
\sqrt[3]{2}
\]

\subsection{Boxed formulas}

The command \cn{boxed} puts a box around its
argument, like \cn{fbox} except that the contents are in math mode:
\begin{equation}
\boxed{\eta \leq C(\delta(\eta) +\Lambda_M(0,\delta))}
\end{equation}
\begin{verbatim}
  \boxed{\eta \leq C(\delta(\eta) +\Lambda_M(0,\delta))}
\end{verbatim}
If you need to box an equation including the equation number, it may be
difficult, depending on the context; there are some suggestions in the
AMS author FAQ; see the entry outlined in red on the page
\url{https://www.ams.org/faq?faq_id=290}.

%%%%%%%%%%%%%%%%%%%%%%%%%%%%%%%%%%%%%%%%%%%%%%%%%%%%%%%%%%%%%%%%%%%%%%%%

\section{Fractions and related constructions}

\subsection{The \hycn{frac}, \hycn{dfrac}, and
  \hycn{tfrac} commands}

The \cn{frac} command\index{fractions} takes two arguments\mdash
numerator and denominator\mdash and typesets them in normal fraction
form. Use \cn{dfrac} or \cn{tfrac} to overrule \LaTeX{}'s guess about
the proper size to use for the fraction's contents (t = text style, d =
display style).
\begin{equation}
\frac{1}{k}\log_2 c(f),\quad\dfrac{1}{k}\log_2 c(f),\quad\tfrac{1}{k}\log_2 c(f)
\end{equation}
\begin{verbatim}
\begin{equation}
\frac{1}{k}\log_2 c(f),\quad\dfrac{1}{k}\log_2 c(f),
    \quad\tfrac{1}{k}\log_2 c(f)
\end{equation}
\end{verbatim}
\begin{equation}
\Re{z} =\frac{n\pi \dfrac{\theta +\psi}{2}}{
        \left(\dfrac{\theta +\psi}{2}\right)^2 + \left( \dfrac{1}{2}
        \log \left\lvert\dfrac{B}{A}\right\rvert\right)^2}.
\end{equation}
\begin{verbatim}
\begin{equation}
\Re{z} =\frac{n\pi \dfrac{\theta +\psi}{2}}{
        \left(\dfrac{\theta +\psi}{2}\right)^2 + \left( \dfrac{1}{2}
        \log \left\lvert\dfrac{B}{A}\right\rvert\right)^2}.
\end{equation}
\end{verbatim}

\subsection{The \hycn{binom}, \hycn{dbinom}, and
        \hycn{tbinom} commands}

For binomial expressions\index{binomials} such as $\binom{n}{k}$
there are \cn{binom}, \cn{dbinom} and \cn{tbinom} commands:
\begin{equation}
2^k-\binom{k}{1}2^{k-1}+\binom{k}{2}2^{k-2}
\end{equation}
\begin{verbatim}
2^k-\binom{k}{1}2^{k-1}+\binom{k}{2}2^{k-2}
\end{verbatim}

\subsection{The \hycn{genfrac} command}

The capabilities of \cn{frac}, \cn{binom}, and their variants are
subsumed by a generalized fraction command \cn{genfrac} with six
arguments. The last two correspond to \cn{frac}'s numerator and
denominator; the first two are optional delimiters (as seen in
\cn{binom}); the third is a line thickness override (\cn{binom} uses
this to set the fraction line thickness to 0 pt\mdash i.e., invisible);
and the fourth argument is a mathstyle override: integer values
0\ndash 3 select, respectively, \cn{displaystyle}, \cn{textstyle},
\cn{scriptstyle}, and \cn{scriptscriptstyle}. If the third argument is
left empty, the line thickness defaults to ``normal''.

\begin{cmdspec}[25em]
\string\genfrac \ma{left-delim} \ma{right-delim} \ma{thickness}
\ma{mathstyle} \ma{numerator} \ma{denominator}
\end{cmdspec}
To illustrate, here is how \cn{frac}, \cn{tfrac}, and
\cn{binom} might be defined.
\begin{verbatim}
\newcommand{\frac}[2]{\genfrac{}{}{}{}{#1}{#2}}
\newcommand{\tfrac}[2]{\genfrac{}{}{}{1}{#1}{#2}}
\newcommand{\binom}[2]{\genfrac{(}{)}{0pt}{}{#1}{#2}}
\end{verbatim}

\begin{notes}
  \singlenote For technical reasons, using the primitive fraction
  commands \cn{over}, \cn{atop}, \cn{above} in a \LaTeX{} document is
  not recommended (see, e.g., \url{https://www.ams.org/faq?faq\_id=288},
  the entry outlined in red).
\end{notes}

\subsection{Continued fractions}

The continued fraction\index{continued fractions}
\begin{equation}
\cfrac{1}{\sqrt{2}+
 \cfrac{1}{\sqrt{2}+
  \cfrac{1}{\sqrt{2}+\cdots
}}}
\end{equation}
can be obtained by typing
{\samepage
\begin{verbatim}
\cfrac{1}{\sqrt{2}+
 \cfrac{1}{\sqrt{2}+
  \cfrac{1}{\sqrt{2}+\dotsb
}}}
\end{verbatim}
}% End of \samepage
This produces better-looking results than straightforward use of
\cn{frac}. Left or right placement of any of the numerators is
accomplished by using \cn{cfrac}\verb|[l]| or \cn{cfrac}\verb|[r]| instead of
\cn{cfrac}.

%%%%%%%%%%%%%%%%%%%%%%%%%%%%%%%%%%%%%%%%%%%%%%%%%%%%%%%%%%%%%%%%%%%%%%%%

\section{Delimiters}\label{delim}

\subsection{Delimiter sizes}\label{bigdel}

Unless you indicate otherwise, delimiters in math formulas will remain
at the standard size regardless of the height of the enclosed material.
To get larger sizes, you can either select a particular size using a
\cn{big...} prefix (see below), or you can use \cn{left} and \cn{right}
prefixes for autosizing.

The automatic delimiter sizing done by \cn{left} and \cn{right} has two
limitations: first, it is applied mechanically to produce delimiters
large enough to encompass the largest contained item, and second, the
range of sizes has fairly large quantum jumps. This means that an
expression that is infinitesimally too large for a given delimiter size
will get the next larger size, a jump of 6pt or so (3pt top and bottom)
in normal-sized text. There are two or three situations where the
delimiter size is commonly adjusted. These adjustments are
done using the following commands:
\begin{center}\begin{tabular}{l|llllll}
Delimiter&
  no size& \ncn{left}& \ncn{bigl}& \ncn{Bigl}& \ncn{biggl}& \ncn{Biggl}\\
size&
  specified& \ncn{right}& \ncn{bigr}& \ncn{Bigr}& \ncn{biggr}& \ncn{Biggr}\\[4pt]
%\hline\omit\rule{0pt}{1ex}\\
\hline\omit\rule{0pt}{1ex}\\[-1ex]
Result $\vphantom{\Bigg|^{\frac{1}{2}}}$ & % force height to avoid gap in vertical
  $\displaystyle(b)(\frac{c}{d})$&
  $\displaystyle\left(b\right)\left(\frac{c}{d}\right)$&
  $\displaystyle\bigl(b\bigr)\bigl(\frac{c}{d}\bigr)$&
  $\displaystyle\Bigl(b\Bigr)\Bigl(\frac{c}{d}\Bigr)$&
  $\displaystyle\biggl(b\biggr)\biggl(\frac{c}{d}\biggr)$&
  $\displaystyle\Biggl(b\Biggr)\Biggl(\frac{c}{d}\Biggr)$
\end{tabular}\end{center}
The first kind of adjustment is done for cumulative operators with
limits, such as summation signs. With \cn{left} and \cn{right} the
delimiters usually turn out larger than necessary, and using the
\verb|Big| or \verb|bigg| sizes\index{big@\cn{big}, \cn{Big}, \cn{bigg},
  \dots\ delimiters} instead gives better results:
\begin{equation*}
\left[\sum_i a_i\left\lvert\sum_j x_{ij}\right\rvert^p\right]^{1/p}
\quad\text{versus}\quad
\biggl[\sum_i a_i\Bigl\lvert\sum_j x_{ij}\Bigr\rvert^p\biggr]^{1/p}
\end{equation*}
\begin{verbatim}
\biggl[\sum_i a_i\Bigl\lvert\sum_j x_{ij}\Bigr\rvert^p\biggr]^{1/p}
\end{verbatim}
The second kind of situation is clustered pairs of delimiters, where
\cn{left} and \cn{right} make them all the same size (because that is
adequate to cover the encompassed material), but what you really want
is to make some of the delimiters slightly larger to make the nesting
easier to see.
\begin{equation*}
\left((a_1 b_1) - (a_2 b_2)\right)
\left((a_2 b_1) + (a_1 b_2)\right)
\quad\text{versus}\quad
\bigl((a_1 b_1) - (a_2 b_2)\bigr)
\bigl((a_2 b_1) + (a_1 b_2)\bigr)
\end{equation*}
\begin{verbatim}
\left((a_1 b_1) - (a_2 b_2)\right)
\left((a_2 b_1) + (a_1 b_2)\right)
\quad\text{versus}\quad
\bigl((a_1 b_1) - (a_2 b_2)\bigr)
\bigl((a_2 b_1) + (a_1 b_2)\bigr)
\end{verbatim}
The third kind of situation is a slightly oversize object in running
text, such as $\left\lvert\frac{b'}{d'}\right\rvert$ where the
delimiters produced by \cn{left} and \cn{right} cause too much line
spreading. In that case \ncn{bigl} and \ncn{bigr}\index{big@\cn{big},
\cn{Big}, \cn{bigg}, \dots\ delimiters} can be used to produce
delimiters that are larger than the base size but still able to
fit within the normal line spacing:
$\bigl\lvert\frac{b'}{d'}\bigr\rvert$.

The \pkg{mathtools} package provides a feature \cn{DeclarePairedDelimiter}
that can simplify sizing; see the package documentation for details.

\subsection{Vertical bar notations}\label{verts}

The use of the \verb'|' character to produce paired delimiters is not
recommended. There is an ambiguity about the directionality of the
symbol that will in rare cases produce incorrect spacing\mdash e.g.,
\verb'|k|=|-k|' produces $|k|=|-k|$, and \verb'|\sin x|' produces $|\sin x|$
instead of the correct $\lvert\sin x\rvert$. Using \cn{lvert} for a \qq{left
vert bar} and \cn{rvert} for a \qq{right vert bar} whenever they are
used in pairs will prevent this problem; compare $\lvert -k\rvert$,
produced by \verb'\lvert -k\rvert'. For double bars there are analogous
\cn{lVert}, \cn{rVert} commands. Recommended practice is to define
suitable commands in the document preamble for any paired-delimiter use
of vert bar symbols:
\begin{verbatim}
\providecommand{\abs}[1]{\lvert#1\rvert}
\providecommand{\norm}[1]{\lVert#1\rVert}
\end{verbatim}
whereupon \verb|\abs{z}| would produce $\lvert z\rvert$ and
\verb|\norm{v}| would produce $\lVert v\rVert$.

%%%%%%%%%%%%%%%%%%%%%%%%%%%%%%%%%%%%%%%%%%%%%%%%%%%%%%%%%%%%%%%%%%%%%%%%

\section{The \hycn{text} command}

The main use of the command \cn{text} is for words or phrases\index{text
  fragments inside math} in a display. It is similar to \cn{mbox} in its
effects but, unlike \cn{mbox}, automatically produces subscript-size
text if used in a subscript.
\begin{equation}
f_{[x_{i-1},x_i]} \text{ is monotonic,}
\quad i = 1,\dots,c+1
\end{equation}
\begin{verbatim}
f_{[x_{i-1},x_i]} \text{ is monotonic,}
\quad i = 1,\dots,c+1
\end{verbatim}

\subsection{\hycn{mod} and its relatives}

Commands \cn{mod}, \cn{bmod}, \cn{pmod}, \cn{pod} deal with the special
spacing conventions of \qq{mod} notation. \cn{mod} and \cn{pod} are
variants of \cn{pmod} preferred by some authors; \cn{mod} omits the
parentheses, whereas \cn{pod} omits the \qq{mod} and retains the
parentheses.
\begin{equation}
\gcd(n,m\bmod n) ;\quad x\equiv y\pmod b
;\quad x\equiv y\mod c ;\quad x\equiv y\pod d
\end{equation}
\begin{verbatim}
\gcd(n,m\bmod n) ;\quad x\equiv y\pmod b
;\quad x\equiv y\mod c ;\quad x\equiv y\pod d
\end{verbatim}

%%%%%%%%%%%%%%%%%%%%%%%%%%%%%%%%%%%%%%%%%%%%%%%%%%%%%%%%%%%%%%%%%%%%%%%%

\section{Integrals and sums}

\subsection{Altering the placement of limits}

The limits on integrals, sums, and similar symbols are placed either to
the side of or above and below the base symbol, depending on convention
and context. \lat/ has rules for automatically choosing one or the
other, and most of the time the results are satisfactory. In the event
they are not, there are three \lat/ commands that can be used to
influence the placement of the limits: \cn{limits}, \cn{nolimits},
\cn{displaylimits}. Compare
\begin{center}
\begin{minipage}{.4\columnwidth}
\[\int_{\abs{x-x_z(t)}<X_0} z^6(t)\phi(x)\]
\begin{verbatim}
\int_{\abs{x-x_z(t)}<X_0} ...
\end{verbatim}
\end{minipage}\quad
and\quad
\begin{minipage}{.5\columnwidth}
\[\int\limits_{\abs{x-x_z(t)}<X_0} z^6(t)\phi(x)\]
\begin{verbatim}
\int\limits_{\abs{x-x_z(t)}<X_0} ...
\end{verbatim}
\end{minipage}
\end{center}
The \cn{limits} command should follow immediately after the base symbol
to which it applies, and its meaning is: shift the following subscript
and/or superscript to the limits position, regardless of the usual
convention for this symbol. \cn{nolimits} means to shift them to the
side instead, and \cn{displaylimits}, which might be used in defining
a new kind of base symbol, means to use standard positioning as for the
\cn{sum} command.

See also the description of the options \opt{intlimits} and
\opt{nosumlimits} in \cite{amsldoc}.

\subsection{Multiple integral signs}

\cn{iint}, \cn{iiint}, and \cn{iiiint} give multiple integral
signs\index{integrals!multiple} with the spacing between them nicely
adjusted, in both text and display style. \cn{idotsint} is an extension
of the same idea that gives two integral signs with dots between them.
Notice the use of thin space (\cn{,})\index{thin space (\protect\cn{,})}
before $dx$ and friends to clarify the meaning.
\begin{gather}
\iint\limits_A f(x,y)\,dx\,dy\qquad\iiint\limits_A
f(x,y,z)\,dx\,dy\,dz\\
\iiiint\limits_A
f(w,x,y,z)\,dw\,dx\,dy\,dz\qquad\idotsint\limits_A f(x_1,\dots,x_k)
\end{gather}
\begin{verbatim}
\iint\limits_A f(x,y)\,dx\,dy\qquad\iiint\limits_A
f(x,y,z)\,dx\,dy\,dz\\
\iiiint\limits_A
f(w,x,y,z)\,dw\,dx\,dy\,dz\qquad\idotsint\limits_A f(x_1,\dots,x_k)
\end{verbatim}

\subsection{Multiline subscripts and superscripts}

The \cn{substack} command can be used to produce a multiline subscript
or superscript:\index{subscripts and superscripts!multiline}\relax
\index{superscripts|see{subscripts and superscripts}} for example
\begin{center}
\begin{tabular}{ll}
\begin{minipage}[t]{.6\columnwidth}
\begin{verbatim}
\sum_{\substack{
         0\le i\le m\\
         0<j<n}}
  P(i,j)
\end{verbatim}
\end{minipage}
&
$\displaystyle
\sum_{\substack{0\le i\le m\\ 0<j<n}} P(i,j)$
\end{tabular}
\end{center}

\subsection{The \hycn{sideset} command}\label{sideset}

There's also a command called \cn{sideset}, for a rather special
purpose: putting symbols at the subscript and
superscript\index{subscripts and superscripts!on sums} corners of a
symbol like $\sum$ or $\prod$. \emph{Note: The
  \cn{sideset} command is only designed for use with large operator
  symbols; with ordinary symbols the results are unreliable.}
With \cn{sideset}, you can write
\begin{center}
\begin{tabular}{ll}
\begin{minipage}[t]{.6\columnwidth}
\begin{verbatim}
\sideset{}{'}
  \sum_{n<k,\;\text{$n$ odd}} nE_n
\end{verbatim}
\end{minipage}
&$\displaystyle
\sideset{}{'}\sum_{n<k,\;\text{$n$ odd}} nE_n
$
\end{tabular}
\end{center}
The extra pair of empty braces is explained by the fact that
\cn{sideset} has the capability of putting an extra symbol or symbols at
each corner of a large operator; to put an asterisk at each corner of a
product symbol, you would type
\begin{center}
\begin{tabular}{ll}
\begin{minipage}[t]{.6\columnwidth}
\begin{verbatim}
\sideset{_*^*}{_*^*}\prod
\end{verbatim}
\end{minipage}
&$\displaystyle
\sideset{_*^*}{_*^*}\prod
$
\end{tabular}
\end{center}

%%%%%%%%%%%%%%%%%%%%%%%%%%%%%%%%%%%%%%%%%%%%%%%%%%%%%%%%%%%%%%%%%%%%%%%%

\section{Changing the size of elements in a formula}

The \lat/ mechanisms for changing font size inside a math formula are
completely different from the ones used outside math formulas. If you
try to make something larger in a formula with one of the text commands
such as \cn{large} or \cn{huge}:
\[\text{\large \#}\qquad\verb'{\large \#}'\]
you will get a warning message
\begin{verbatim}
Command \large invalid in math mode
\end{verbatim}
Such an attempt, however, often indicates a misunderstanding of how
\lat/ math symbols work. If you want a \# symbol analogous to a
summation sign in its typographical properties, then in principle the
best way to achieve that is to define it as a symbol of type ``mathop''
with the standard \lat/ \cn{DeclareMathSymbol} command (see
\cite{fntguide}). (This entails, however, getting hold of a math font
with a suitable text-size\slash display-size pair, which may not be so
easy.)

Consider the expression:
\[\frac{\sum_{n > 0} z^n}{\prod_{1\leq k\leq n} (1-q^k)}
\qquad\begin{minipage}{.5\columnwidth}
\begin{verbatim}
\frac{\sum_{n > 0} z^n}
     {\prod_{1\leq k\leq n} (1-q^k)}
\end{verbatim}
\end{minipage}
\]
Using \cn{dfrac} instead of \cn{frac} wouldn't change anything in this
case; if you want the sum and product symbols to appear full size, you
need the \cn{displaystyle} command:
\[
\frac{{\displaystyle\sum_{n > 0} z^n}}
     {{\displaystyle\prod_{1\leq k\leq n} (1-q^k)}}
\qquad\begin{minipage}{.7\columnwidth}
\begin{verbatim}
\frac{{\displaystyle\sum_{n > 0} z^n}}
     {{\displaystyle\prod_{1\leq k\leq n} (1-q^k)}}
\end{verbatim}
\end{minipage}
\]
And if you want full-size symbols but with limits on the side, use
the \cn{nolimits} command also:
\[
\frac{{\displaystyle\sum\nolimits_{n> 0} z^n}}
  {{\displaystyle\prod\nolimits_{1\leq k\leq n} (1-q^k)}}
\qquad\begin{minipage}{.76\columnwidth}
\begin{verbatim}
\frac{{\displaystyle\sum\nolimits_{n> 0} z^n}}
  {{\displaystyle\prod\nolimits_{1\leq k\leq n} (1-q^k)}}
\end{verbatim}
\end{minipage}
\]
There are similar commands \cn{textstyle}, \cn{scriptstyle}, and
\cn{scriptscriptstyle}, to force \lat/ to use the symbol size and
spacing that would be applied in (respectively) inline math, first-order
subscript, or second-order subscript, even when the current context
would normally yield some other size.

\textbf{Note:} These commands belong to a special class of
commands referred to in the \lat/ book as ``declarations''. In
particular, notice where the braces fall that delimit the effect of the
command:
\begin{center}
\textbf{Right: } \verb'{\displaystyle ...}'
\qquad\qquad\textbf{Wrong: } \verb'\displaystyle{...}'
\end{center}

%%%%%%%%%%%%%%%%%%%%%%%%%%%%%%%%%%%%%%%%%%%%%%%%%%%%%%%%%%%%%%%%%%%%%%%%

\section{Other packages of interest}
\label{other-packages}

Many other \LaTeX{} packages that address some aspect of mathematical
formulas are available from CTAN (the Comprehensive \TeX{} Archive
Network). To recommend a few examples:
\begin{description}
\raggedright
\item[mathtools] Additional features extending \pkg{amsmath}; loads
  \pkg{amsmath}.
\item[amsthm] General theorem and proof setup.
\item[amsfonts] Defines \cn{mathbb} and \cn{mathfrak}, and provides access
  to many additional symbols (without names; \pkg{amssymb} provides the
  names).
\item[accents] Under accents and accents using arbitrary symbols.
\item[bm] Bold math package, provides a more general and more robust
  implementation of \cn{boldsymbol}.
\item[mathrsfs] Ralph Smith's Formal Script, font setup.
\item[cases] Apply a large brace to two or more equations without
  losing the individual equation numbers.
\item[delarray] Delimiters spanning multiple rows of an array.
%%\item[kuvio] Commutative diagrams and other diagrams. % not in TeX Live
\item[xypic] Commutative diagrams and other diagrams.
\item[TikZ] Comprehensive graphical facilities, including features for
  drawing diagrams.
\end{description}

The \TeX{} Catalogue,\\
\null\hspace{2\parindent}
\url{http://mirror.ctan.org/help/Catalogue/alpha.html},\\
is a good place to look if you know a package's name.

\medskip
Questions and answers on specific \TeX-related topics are the
\emph{raison d'\^etre} of this forum:\\
\null\hspace{2\parindent}
\url{https://tex.stackexchange.com/questions}\\
Check the archives for existing answers; pointers to selected topics
may expedite your search:\\
\null\hspace{2\parindent}
\url{https://tex.meta.stackexchange.com/a/2425#2425}\\
If nothing useful turns up, ask your own question.

%\newpage

%%%%%%%%%%%%%%%%%%%%%%%%%%%%%%%%%%%%%%%%%%%%%%%%%%%%%%%%%%%%%%%%%%%%%%%%

\section{Other documentation of interest}

\begin{thebibliography}{AMUG}
\raggedright

\bibitem[AMUG]{amsldoc} American Mathematical Society and the \LaTeX3 Project:
  \emph{User's Guide for the \textnormal{\ttfamily amsmath} package},
  Version~2.$+$,
  \url{http://mirror.ctan.org/macros/latex/required/amsmath/amsldoc.tex} and
  \url{http://mirror.ctan.org/macros/latex/required/amsmath/amsldoc.pdf},
  2017.

\bibitem[AFUG]{amsfndoc} American Mathematical Society:
  \emph{User's Guide, AMSFonts}, 
  \url{http://mirror.ctan.org/fonts/amsfonts/amsfndoc.pdf}, 2002.

\bibitem[CLSL]{comprehensive} Scott Pakin:
  \emph{The Comprehensive \LaTeX{} Symbol List},
  \url{http://mirror.ctan.org/tex-archive/info/symbols/comprehensive/},
  January 2017.  Raw font tables, without symbol names, are shown
  alphabetically by font name in the \fn{rawtables*.pdf} files in the
  same area of CTAN and from \TeX\,Live with \texttt{texdoc rawtables}.

\bibitem[Lam]{lamport} Leslie Lamport: \emph{\LaTeX{}: A document
    preparation system}, 2nd edition, Addison-Wesley, 1994.

\bibitem[LC]{companion} Frank Mittelbach and Michel Goossens,
  with Johannes Braams, David Carlisle, and Chris Rowley:
  \emph{The \LaTeX{} Companion}, 2nd edition, Addison-Wesley, 2004.

\bibitem[LFG]{fntguide} \LaTeX3 Project Team: \emph{\LaTeXe{} font
  selection}, % \fn{fntguide.tex}, November 2005.
  \url{http://mirror.ctan.org/macros/latex/doc/fntguide.pdf}, 2005.

\bibitem[LGC]{graphics-companion} Michel Goossens, Frank Mittelbach,
  Sebastian Rahtz, Denis Roegel, and Herbert~Vo\ss:
  \emph{The \LaTeX{} Graphics Companion}, 2nd edition, Addison-Wesley, 2008.

\bibitem[LGG]{grfguide} D.~P. Carlisle, \LaTeX3 Project:
  \emph{Packages in the `graphics' bundle}, %\fn{grfguide.tex}, 2017.
  \url{http://mirror.ctan.org/macros/latex/required/graphics/grfguide.pdf},
  2017.

\bibitem[LUG]{usrguide} \LaTeX3 Project Team: \emph{\LaTeXe~for
  authors}, % \fn{usrguide.tex}, 2015.
  \url{http://mirror.ctan.org/macros/latex/doc/usrguide.pdf}, 2015.

\bibitem[MML]{gratzer} George Gr\"atzer: \textit{More Math into \LaTeX},
   5th edition, Springer, New York, 2016.

\bibitem[UCM]{uc-math} Will Robertson: \emph{Every symbol
  \textup{(}most symbols\textup{)} defined by \pkg{unicode-math}},
  \url{http://mirror.ctan.org/macros/latex/contrib/unicode-math/unimath-symbols.pdf}, 2017; and\\
  Will Robertson, Philipp Stephani, Joseph Wright, and Khaled Hosny:
  \emph{Experimental Unicode mathematical typesetting: The \pkg{unicode-math}
  package},
  \url{http://mirror.ctan.org/macros/latex/contrib/unicode-math/unicode-math.pdf}, 2017.

\end{thebibliography}

\end{document}
