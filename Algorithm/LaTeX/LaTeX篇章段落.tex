%!TEX program = xelatex 
\documentclass{article} 
\usepackage{xunicode, xltxtra, ctex, amsmath} 
\usepackage{lmodern}
\usepackage[a4paper,left=3cm,right=3cm,top=2cm,bottom=2cm]{geometry} % paperwidth=11cm,scale=0.8

\title{My First \LaTeX \quad Document} 
\author{Wilson79} 
\date{\today}

\begin{document} 
    \tableofcontents % 产生文档目录
    \thispagestyle{empty} %目录页不显示页码
    \newpage
    \setcounter{page}{1} % 从下面开始编页码
    
    \maketitle %{第一章} 

    %\chapter{绪论} % 章节命令 需要选book类,另外此时subsubsection无效
    \section{Introduction} 
    准备好了!本章将见识到 LATEX 闻名的强项——排版数学公式。当然你得注意了, 本章的内容只是一点皮毛,虽然对大多数人来说已经够用了,但是如果不能解决你的问 题的话也不要大惊小怪,求助于搜索引擎或者有经验的人不失为一个好办法。
    
    本章介绍一些特色的 LATEX 辅助功能。前两个功能 BIBTEX 和 makeindex 依靠一些 辅助程序自动生成参考文献、索引等;之后的使用颜色、超链接等则令我们生成美观易 用的电子文档。 

    \section{Experimental method} %小节

    \section{Experimental result}
    \subsection{date} %子小节
    \subsection{graph} %子小节
    \subsubsection{experiment condition} %子小节
    \subsubsection{experimental process} %子小节

    \section{Conclusion}


\end{document}