\section{表与图}这节用来展示表格与图片的插入。

\subsection{表格}
\par 本来LaTeX里表格的变化是非常多的,但鉴于学校要求用三线式,问题反而简单了。以下是一个例子:
\begin{table}[htbp]\center
    \caption{示例表格\\Example Table}
    \begin{tabular}{lcccccl}
     \toprule
     。。 & 。。 & 。。 & 。。 & 。。& 。。 & 。。\\
     \midrule
    。。 & 。。 & 。。 & 。。 & 。。& 。。 & 。。\\
    。。 & 。。 & 。。 & 。。 & 。。& 。。 & 。。\\
    。。 & 。。 & 。。 & 。。 & 。。& 。。 & 。。\\
    。。 & 。。 & 。。 & 。。 & 。。& 。。 & 。。\\
    。。 & 。。 & 。。 & 。。 & 。。& 。。 & 。。\\
     \bottomrule
    \end{tabular}
   \end{table}
如果你有使用更复杂的表格的需求,请自行查资料完成。

\subsection{插图}
由于这份模板不考虑多栏排版,所以格式要求中所述的半栏图大小要求我们不作演示。以下是一个通栏图的演示:
\begin{figure}[H]
    \centering
    \includegraphics[width=100mm]{example-image}
    \caption{图片测试(最小宽度)\\Image test (Minimal width)}
  \end{figure}

\begin{figure}[H]
    \centering
    \includegraphics[width=130mm]{example-image}
    %\includegraphics[width=130mm]{./figures/你自己的图像文件}
    \caption{图片测试(最大宽度)\\Image test (Maximal width)}
\end{figure}
\par 注意:这里为了减少图片上下的空白,使用了float宏包。