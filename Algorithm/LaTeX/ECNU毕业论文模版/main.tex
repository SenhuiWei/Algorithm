% !TEX root
% !TEX program = xelatex
% !BIB program = biber

\def \PrintMode{} %在使用电子版论文时,请将此行注释。在打印纸质论文时,请保持本行命令不被注释,然后打印时选择双面打印即可。

%用来控制是否启动打印模式的宏,请勿改动。
\ifx \PrintMode \undefined
    \def \SideMode{oneside}
    \def \ClearPageStyle{\clearpage}
\else
    \def \SideMode{twoside}
    \def \ClearPageStyle{\cleardoublepage}
\fi

\documentclass[a4paper,\SideMode,UTF8]{article} %A4纸,UTF-8

\usepackage[thmmarks,hyperref]{ntheorem} %定义命令环境使用的宏包
\usepackage[heading,zihao=-4]{ctex} %用来提供中文支持
\usepackage{amsmath,amssymb} %数学符号等相关宏包
\usepackage{graphicx} %插入图片所需宏包
\usepackage{xspace} %提供一些好用的空格命令
\usepackage{tikz-cd} %画交换图需要的宏包
\usepackage{url} %更好的超链接显示
\usepackage{array,booktabs} %表格相关的宏包
\usepackage{caption} %实现图片的多行说明
\usepackage{float} %图片与表格的更好排版
\usepackage{ulem} %更好的下划线
\usepackage[top=2.5cm, bottom=2.0cm, left=3.0cm, right=2.0cm]{geometry} %设置页边距

\usepackage{fontspec} %设置字体需要的宏包

%设置西文字体为Times New Roman,如果没有则以开源近似字体代替
\IfFontExistsTF{Times New Roman}{
	\setmainfont{Times New Roman}
}{
	\usepackage{newtxtext,newtxmath}
}

%设置文档中文字体。优先次序:中易 > Adobe > 华文(Mac) > Fandol
\IfFontExistsTF{SimSun}{
	\setCJKmainfont[AutoFakeBold=2,ItalicFont=KaiTi]{SimSun}
}{
	\IfFontExistsTF{AdobeSongStd-Light}{
		\setCJKmainfont[AutoFakeBold=2,ItalicFont=AdobeKaitiStd-Regular]{AdobeSongStd-Light}
	}{
		\IfFontExistsTF{STSong}{
			\setCJKmainfont[AutoFakeBold=2,BoldFont=STHeiti,ItalicFont=STKaiti]{STSong}
		}{
			\setCJKmainfont[AutoFakeBold=2,ItalicFont=FandolKai-Regular]{FandolSong-Regular}
		}
	}
}
\IfFontExistsTF{SimHei}{
	\setCJKsansfont[AutoFakeBold=2]{SimHei}
}{
	\IfFontExistsTF{AdobeHeitiStd-Regular}{
		\setCJKsansfont[AutoFakeBold=2]{AdobeHeitiStd-Regular}
	}{
		\IfFontExistsTF{STHeiti}{
			\setCJKsansfont [AutoFakeBold=2]{STHeiti}
		}{
			\setCJKsansfont[AutoFakeBold=2]{FandolHei-Regular}
		}
	}
}


\IfFileExists{zhlinepip.sty}{
	%Microsoft Word 样式的1.5倍行距(按中易字体计算)
	\usepackage[
		restoremathleading=false,
		UseMSWordMultipleLineSpacing,
		MSWordLineSpacingMultiple=1.5
	]{zhlinepip}
}{
	\linespread{1.621} %1.5倍行距
}

\showboxdepth=5
\showboxbreadth=5

%设置各级系统的编号格式
\setcounter{secnumdepth}{5}
\ctexset { section = { name={,.},number={\arabic{section}},format={\sffamily \zihao {-4}} } }
\ctexset { subsection = { name={,},number={\arabic{section}.\arabic{subsection}},format={\sffamily \zihao {-4}} } }
\ctexset { subsubsection = { name={,},number={\arabic{section}.\arabic{subsection}.\arabic{subsubsection}},format={\sffamily \zihao {-4}},indent=2em } }
\ctexset { paragraph = { name={,},number={\arabic{section}.\arabic{subsection}.\arabic{subsubsection}.\arabic{paragraph}},format={\sffamily \zihao {-4}},indent=4em } }
\ctexset { subparagraph = { name={,)},number={\arabic{subparagraph}},format={\sffamily \zihao {-4}},indent=6em } }

\usepackage[bottom,perpage]{footmisc}               %脚注,显示在每页底部,编号按页重置
\renewcommand*{\footnotelayout}{\zihao{-5}\rmfamily}  %设置脚注为小五号宋体
\renewcommand{\thefootnote}{\textcircled{\arabic{footnote}}}    %设置脚注标记为①,②,...

%设置页眉页脚
\usepackage{fancyhdr}
\lhead{华东师范大学学士学位论文}
\chead{}
\rhead{\TitleCHS}
\lfoot{}
\cfoot{\thepage}
\rfoot{}

\usepackage{xcolor} %彩色的文字

\usepackage[hidelinks]{hyperref} %各种超链接必备
\usepackage{cleveref} %交叉引用

%设置尾注
\usepackage{endnotes}
\renewcommand{\enotesize}{\zihao{-5}}
\renewcommand{\notesname}{\sffamily \zihao {-4} 尾注}
\renewcommand\enoteformat{
	\raggedright
	\leftpip=1.8em
	\makebox[0pt][r]{\theenmark. \rule{0pt}{\dimexpr\ht\strutbox+\baselinepip}}
}
\renewcommand\makeenmark{\textsuperscript{[尾注\theenmark]}}
\usepackage{footnotebackref}

%定义证明与解环境
\theoremstyle{nonumberplain}
\theorembodyfont{\upshape}
\theoremseparator{:}
\theoremsymbol{\ensuremath{\square}}
\newtheorem{proof}{\bfseries \sffamily 证明}
\theoremsymbol{\ensuremath{\blacksquare}}
\newtheorem{solution}{\bfseries \sffamily 解}

%定义各种常用环境
\theoremstyle{plain}
\theoremseparator{.}
\theorembodyfont{\upshape}
\theoremsymbol{}
\newtheorem{theorem}{\bfseries \sffamily 定理}[section]
\renewtheorem*{theorem*}{\bfseries \sffamily 定理}
\newtheorem{lemma}[theorem]{\bfseries \sffamily 引理}
\renewtheorem*{lemma*}{\bfseries \sffamily 引理}
\newtheorem{corollary}[theorem]{\bfseries \sffamily 推论}
\renewtheorem*{corollary*}{\bfseries \sffamily 推论}
\newtheorem{definition}[theorem]{\bfseries \sffamily 定义}
\renewtheorem*{definition*}{\bfseries \sffamily 定义}
\newtheorem{conjecture}[theorem]{\bfseries \sffamily 猜想}
\renewtheorem*{conjecture*}{\bfseries \sffamily 猜想}
\newtheorem{problem}[theorem]{\bfseries \sffamily 问题}
\renewtheorem*{problem*}{\bfseries \sffamily 问题}
\newtheorem{proposition}[theorem]{\bfseries \sffamily 命题}
\renewtheorem*{proposition*}{\bfseries \sffamily 命题}
\newtheorem{remark}[theorem]{\bfseries \sffamily 注记}
\renewtheorem*{remark*}{\bfseries \sffamily 注记}
\newtheorem{example}[theorem]{\bfseries \sffamily 例}
\renewtheorem*{example*}{\bfseries \sffamily 例}

%设置各种常用环境的交叉引用格式
\crefformat{theorem}{#2\bfseries{\sffamily 定理} #1#3}
\crefformat{lemma}{#2\bfseries{\sffamily 引理} #1#3}
\crefformat{corollary}{#2\bfseries{\sffamily 推论} #1#3}
\crefformat{definition}{#2\bfseries{\sffamily 定义} #1#3}
\crefformat{conjecture}{#2\bfseries{\sffamily 猜想} #1#3}
\crefformat{problem}{#2\bfseries{\sffamily 问题} #1#3}
\crefformat{proposition}{#2\bfseries{\sffamily 命题} #1#3}
\crefformat{remark}{#2\bfseries{\sffamily 注记} #1#3}
\crefformat{example}{#2\bfseries{\sffamily 例} #1#3}

%允许公式跨页显示
\allowdisplaybreaks

%屏蔽无关的Warning
\usepackage{silence}
\WarningFilter*{biblatex}{Conflicting options.\MessageBreak'eventdate=iso' requires 'seconds=true'.\MessageBreak Setting 'seconds=true'}

%使用biblatex管理文献,输出格式使用gb7714-2015标准,后端为biber
\usepackage[backend=biber,style=gb7714-2015,hyperref=true]{biblatex}
%将参考文献字体设置为五号
\renewcommand*{\bibfont}{\zihao{5}}

%生成感谢,请勿改动
\newcommand{\makeacknowledgement}{
	\clearpage
	\input{./ending/acknowledgement.tex}
}

%For Algorithm
\usepackage{algorithm,algorithmicx,algpseudocode}
\floatname{algorithm}{算法}
\renewcommand{\algorithmicrequire}{\textbf{输入:}}
\renewcommand{\algorithmicensure}{\textbf{输出:}}

%可能会需要在用自然语言描述算法步骤时使用的宏包
\usepackage{enumitem}

%表格单元格内换行
\newcommand{\tabincell}[2]{\begin{tabular}{@{}#1@{}}#2\end{tabular}}

%设置图、表的编号格式
\renewcommand{\thefigure}{\arabic{section}-\arabic{figure}}
\renewcommand{\thetable}{\arabic{section}-\arabic{table}}
%%每个section开始重置图、表的计数器
\makeatletter
\@addtoreset{table}{section}
\makeatother
\makeatletter
\@addtoreset{figure}{section}
\makeatother

%显示 1、2级标题
\setcounter{tocdepth}{2}

%设置目录字体
\usepackage{tocloft}
\renewcommand{\contentsname}{\centerline{目录}}
\renewcommand{\cftaftertoctitle}{\hfill}
\renewcommand{\cfttoctitlefont}{\sffamily \bfseries \zihao{-3}}
\renewcommand{\cftsubsubsecfont}{\rmfamily}
\renewcommand{\cftsubsecfont}{\rmfamily}
\renewcommand{\cftsecfont}{\rmfamily}
\renewcommand{\cftsecleader}{\cftdotfill{\cftdotsep}}
\renewcommand{\cftsecfont}{}
\renewcommand{\cftsecpagefont}{}

%灵活的行距定义(用于封面)
\usepackage{setspace}
%使用绝对坐标制作封面使用的宏包
\usepackage[absolute,overlay]{textpos}
  \setlength{\TPHorizModule}{1mm}
  \setlength{\TPVertModule}{1mm} %加载各宏包以及本模板的主要设置
\addbibresource{./reference/thesis-ref.bib} %加载bib文件(参考文献)

\begin{document}

\pagestyle{empty} %不对正文前的各页面使用页眉页脚
\newgeometry{top=2.0cm, bottom=2.0cm,left=3.18cm, right=3.18cm} %设置用于首页的页边距

%请不要修改本页的任何代码!
%请不要修改本页的任何代码!
%请不要修改本页的任何代码!
\thispagestyle{empty}
\begin{titlepage}
	\captionsetup{belowpip=0pt}
	\newcommand{\TitleCHS}{ 华东师范大学本科毕业论文\LaTeX 模板} %中文标题

\newcommand{\TitleENG}{ \LaTeX\xspace Template for  Undergraduate Dissertation in ECNU } %英文标题

\newcommand{\Author}{张三} %作者名字

\newcommand{\StudentID}{23333333333} %学号

\newcommand{\Department}{理工学院} %学院

\newcommand{\Major}{进口挖掘机修理} %专业

\newcommand{\Supervisor}{李四} %导师名字

\newcommand{\AcademicTitle}{副工程师} %导师职称

\newcommand{\CompleteYear}{2048} %毕业年份

\newcommand{\CompleteMonth}{3} %毕业月份

\newcommand{\KeywordsCHS}{关键词1,关键词2,关键词3,关键词4,关键词5,关键词6,关键词7 } %中文关键词

\newcommand{\KeywordsENG}{keyword1, keyword2, keyword3, keyword4,keyword5, keyword6, keyword7} %英文关键词

	\renewcommand{\ULthickness}{1.2pt}
	\begin{center}\noindent \bfseries \zihao{4}{\rmfamily{\CompleteYear 届本科生学士学位论文\hfill 学校代码:\uline{10269}}}\end{center}

	\begin{textblock}{146.4}(31.8,31)
		\centering
		\includegraphics{./figures/inner-cover(contains_font).eps}
	\end{textblock}

	\begin{textblock}{146.4}(31.8,125)
		\setstretch{2.0}
		\noindent
		\begin{minipage}[t][8.2cm][c]{\linewidth}
			\begin{center}
				\noindent\textbf{\zihao{1}{\rmfamily{\expandafter\uline\expandafter{\TitleCHS}}}}
			\end{center}
			\begin{center}
				\noindent\textbf{\zihao{1}{\rmfamily{\expandafter\uline\expandafter{\TitleENG}}}}
			\end{center}
		\end{minipage}
	\end{textblock}

	\renewcommand{\ULthickness}{0.4pt}

	\begin{textblock}{146.4}(31.8,208)
		\begin{center}
			\renewcommand{\arraystretch}{0.9}
			\bfseries\zihao{4}\rmfamily
			\begin{tabular}{ l r }
				姓\hfill 名:                   & \underline{{\makebox[6cm][c]{\Author}}}        \\
				学\hfill 号:                   & \underline{{\makebox[6cm][c]{\StudentID}}}     \\
				学\hfill 院:                   & \underline{{\makebox[6cm][c]{\Department}}}    \\
				专\hfill 业:                   & \underline{{\makebox[6cm][c]{\Major}}}         \\
				指\hfill 导\hfill 教\hfill 师: & \underline{{\makebox[6cm][c]{\Supervisor}}}    \\
				职\hfill 称:                   & \underline{{\makebox[6cm][c]{\AcademicTitle}}} \\
			\end{tabular}\\
			\vspace{1em}
			\CompleteYear\hspace*{1em}年\hspace*{1em}\CompleteMonth\hspace*{1em}月
		\end{center}
	\end{textblock}
	
\end{titlepage} %插入内封面
\ClearPageStyle

\restoregeometry
%生成目录
\addtocontents{toc}{\protect\thispagestyle{empty}}
\begin{spacing}{1}
    \tableofcontents
\end{spacing}
\ClearPageStyle
\pagenumbering{Roman}
\thispagestyle{fancy}
\newcommand{\TitleCHS}{ 华东师范大学本科毕业论文\LaTeX 模板} %中文标题

\newcommand{\TitleENG}{ \LaTeX\xspace Template for  Undergraduate Dissertation in ECNU } %英文标题

\newcommand{\Author}{张三} %作者名字

\newcommand{\StudentID}{23333333333} %学号

\newcommand{\Department}{理工学院} %学院

\newcommand{\Major}{进口挖掘机修理} %专业

\newcommand{\Supervisor}{李四} %导师名字

\newcommand{\AcademicTitle}{副工程师} %导师职称

\newcommand{\CompleteYear}{2048} %毕业年份

\newcommand{\CompleteMonth}{3} %毕业月份

\newcommand{\KeywordsCHS}{关键词1,关键词2,关键词3,关键词4,关键词5,关键词6,关键词7 } %中文关键词

\newcommand{\KeywordsENG}{keyword1, keyword2, keyword3, keyword4,keyword5, keyword6, keyword7} %英文关键词

\renewcommand\abstractname{\sffamily\zihao{-3} 摘要}
\phantomsection
\begin{abstract}
	\addcontentsline{toc}{section}{摘要}
	\zihao{5}\rmfamily
	\vspace{\baselinepip}
	\par 这里是中文摘要。这里是中文摘要。这里是中文摘要。这里是中文摘要。这里是中文摘要。这里是中文摘要。这里是中文摘要。这里是中文摘要。这里是中文摘要。这里是中文摘要。这里是中文摘要。这里是中文摘要。这里是中文摘要。这里是中文摘要。这里是中文摘要。这里是中文摘要。这里是中文摘要。这里是中文摘要。这里是中文摘要。
	\par 这里是中文摘要。这里是中文摘要。这里是中文摘要。这里是中文摘要。这里是中文摘要。这里是中文摘要。这里是中文摘要。这里是中文摘要。这里是中文摘要。这里是中文摘要。这里是中文摘要。这里是中文摘要。这里是中文摘要。这里是中文摘要。这里是中文摘要。这里是中文摘要。这里是中文摘要。这里是中文摘要。这里是中文摘要。
	\par 这里是中文摘要。这里是中文摘要。这里是中文摘要。这里是中文摘要。这里是中文摘要。这里是中文摘要。这里是中文摘要。这里是中文摘要。这里是中文摘要。这里是中文摘要。这里是中文摘要。这里是中文摘要。这里是中文摘要。这里是中文摘要。这里是中文摘要。这里是中文摘要。这里是中文摘要。这里是中文摘要。这里是中文摘要。
	\par 这里是中文摘要。这里是中文摘要。这里是中文摘要。这里是中文摘要。这里是中文摘要。这里是中文摘要。这里是中文摘要。这里是中文摘要。这里是中文摘要。这里是中文摘要。这里是中文摘要。这里是中文摘要。这里是中文摘要。这里是中文摘要。这里是中文摘要。这里是中文摘要。这里是中文摘要。这里是中文摘要。这里是中文摘要。
	\newline
	\newline
	{\bfseries \rmfamily\zihao{5} 关键词:} \zihao{5}{\rmfamily \KeywordsCHS}
\end{abstract} %生成中英文摘要及关键词
\ClearPageStyle

\thispagestyle{fancy}
\renewcommand\abstractname{\zihao{-3} Abstract}
\phantomsection
\begin{abstract}
    \addcontentsline{toc}{section}{ABSTRACT}
    \zihao{5}
    \vspace{\baselinepip}
    \par Here is Abstract in English. Here is Abstract in English. Here is Abstract in English. Here is Abstract in English. Here is Abstract in English. Here is Abstract in English. Here is Abstract in English. Here is Abstract in English. Here is Abstract in English. Here is Abstract in English. Here is Abstract in English. Here is Abstract in English. Here is Abstract in English. Here is Abstract in English. Here is Abstract in English. Here is Abstract in English. Here is Abstract in English. Here is Abstract in English. Here is Abstract in English. Here is Abstract in English. Here is Abstract in English. Here is Abstract in English. Here is Abstract in English. 
    \par Here is Abstract in English. Here is Abstract in English. Here is Abstract in English. Here is Abstract in English. Here is Abstract in English. Here is Abstract in English. Here is Abstract in English. Here is Abstract in English. Here is Abstract in English. Here is Abstract in English. Here is Abstract in English. Here is Abstract in English. Here is Abstract in English. Here is Abstract in English. Here is Abstract in English. Here is Abstract in English. Here is Abstract in English. Here is Abstract in English. Here is Abstract in English. Here is Abstract in English. Here is Abstract in English. Here is Abstract in English. Here is Abstract in English. 
    \par Here is Abstract in English. Here is Abstract in English. Here is Abstract in English. Here is Abstract in English. Here is Abstract in English. Here is Abstract in English. Here is Abstract in English. Here is Abstract in English. Here is Abstract in English. Here is Abstract in English. Here is Abstract in English. Here is Abstract in English. Here is Abstract in English. Here is Abstract in English. Here is Abstract in English. Here is Abstract in English. Here is Abstract in English. Here is Abstract in English. Here is Abstract in English. Here is Abstract in English. Here is Abstract in English. Here is Abstract in English. Here is Abstract in English.     
    \par Here is Abstract in English. Here is Abstract in English. Here is Abstract in English. Here is Abstract in English. Here is Abstract in English. Here is Abstract in English. Here is Abstract in English. Here is Abstract in English. Here is Abstract in English. Here is Abstract in English. Here is Abstract in English. Here is Abstract in English. Here is Abstract in English. Here is Abstract in English. Here is Abstract in English. Here is Abstract in English. Here is Abstract in English. Here is Abstract in English. Here is Abstract in English. Here is Abstract in English. Here is Abstract in English. Here is Abstract in English. Here is Abstract in English. 
    \par Here is Abstract in English. Here is Abstract in English. Here is Abstract in English. Here is Abstract in English. Here is Abstract in English. Here is Abstract in English. Here is Abstract in English. Here is Abstract in English. Here is Abstract in English. Here is Abstract in English. Here is Abstract in English. Here is Abstract in English. Here is Abstract in English. Here is Abstract in English. Here is Abstract in English. Here is Abstract in English. Here is Abstract in English. Here is Abstract in English. Here is Abstract in English. Here is Abstract in English. Here is Abstract in English. Here is Abstract in English. Here is Abstract in English. 
    \par Here is Abstract in English. Here is Abstract in English. Here is Abstract in English. Here is Abstract in English. Here is Abstract in English. Here is Abstract in English. Here is Abstract in English. Here is Abstract in English. Here is Abstract in English. Here is Abstract in English. Here is Abstract in English. Here is Abstract in English. Here is Abstract in English. Here is Abstract in English. Here is Abstract in English. Here is Abstract in English. Here is Abstract in English. Here is Abstract in English. Here is Abstract in English. Here is Abstract in English. Here is Abstract in English. Here is Abstract in English. Here is Abstract in English. 
    \par Here is Abstract in English. Here is Abstract in English. Here is Abstract in English. Here is Abstract in English. Here is Abstract in English. Here is Abstract in English. Here is Abstract in English. Here is Abstract in English. Here is Abstract in English. Here is Abstract in English. Here is Abstract in English. Here is Abstract in English. Here is Abstract in English. Here is Abstract in English. Here is Abstract in English. Here is Abstract in English. Here is Abstract in English. Here is Abstract in English. Here is Abstract in English. Here is Abstract in English. Here is Abstract in English. Here is Abstract in English. Here is Abstract in English. 
    \newline
    \newline
    {\bfseries \zihao{5} Keywords:} {\zihao{5} \KeywordsENG}
\end{abstract} %生成中英文摘要及关键词
\ClearPageStyle
\pagenumbering{arabic}
\pagestyle{fancy} %开始使用页眉页脚
\setcounter{page}{1} %论文页码从正文开始记数

\section{章节结构测试}这节用来展示文章的5层结构。事实上,一般来说文章层次在3-4层为宜。在之后的section中,我们会只使用至多3层结构(即,节-小节-子节)来进行各种演示。
 
\subsection{小节标题}这一小节我们介绍这些内容。

\subsubsection{子节标题}这一子节我们介绍这些内容。

\paragraph{段标题}这一段我们介绍这些内容。 

\subparagraph{小段标题}这一小段我们介绍这些内容。 %正文第一章
\section{定理等环境测试}这节用来展示定理,引理等常用论文环境。
 
\subsection{编号环境与不编号环境}

\subsubsection{编号环境}

\begin{theorem}\label{thm2_1}

    设$A,B$是两个实数, 则$2AB\leq 2 A^2+B^2$.
    
\end{theorem}

\begin{proof}
    这里是证明。
\end{proof}

\begin{lemma}[Nakayama引理]\label{lem2_2}
    这是一条引理测试。。。
\end{lemma}

\begin{problem}[连续统假设]是否存在$\mathbb{R}$的子集S使得$card(\mathbb{N})<card(S)<card(\mathbb{R})$?
\end{problem}
\begin{solution}
    不存在。
\end{solution}

\subsubsection{无编号环境}

\begin{theorem*}

    设$A,B$是两个实数, 则$2AB\leq 2 A^2+B^2$.
    
\end{theorem*}

\begin{proof}
    这里是证明。
\end{proof}

\begin{lemma*}[Nakayama引理]
    这是一条引理测试。。。
\end{lemma*}

\begin{problem*}[连续统假设]是否存在$\mathbb{R}$的子集S使得$card(\mathbb{N})<card(S)<card(\mathbb{R})$?
\end{problem*}
\begin{solution}
    不存在。
\end{solution} %正文第二章
\section{公式测试}这节用来展示公式,交换图等。
 
\subsection{行内公式}
典范的同态$\lim_{\leftarrow F} W_r(S)\rightarrow \lim_{\leftarrow F} W_r(S/\pi S )$是同构。

\subsection{整行公式}
$$\mathbb{A}_{inf}=W(S^\flat)\cong \lim_{\leftarrow F} W_r(S)$$

\subsection{多行公式}
\begin{sloppypar}
多行公式的情况非常多,对齐与换行的要求也各不相同。所以选择合适的环境非常重要。这份文档里无法涵盖所有情况,所以提供一个教程用以参考:\url{http://blog.csdn.net/yanxiangtianji/article/details/54767265}
\end{sloppypar}



\subsubsection{align环境}
\begin{align*}
    \operatorname{E} (Z_{n+1} - Z_n | X_1,..., X_n)
    &= \operatorname{E} (S_{n+1}^2 - (n+1) \sigma^2 - S_n^2 + n \sigma^2 | X_1,..., X_n) \\
    &= \operatorname{E} (S_{n+1}^2 - S_n^2 - (n+1) \sigma^2 + n \sigma^2 | X_1,..., X_n) \\
    &= \operatorname{E} (X_{n+1}(X_{n+1} + 2\sum_{i=1}^n X_i) - \sigma^2 | X_1,..., X_n) \\
    &= \operatorname{E} (X_{n+1}X_{n+1})
       + 2\operatorname{E} (X_{n+1}) \sum_{i=1}^n X_i - \sigma^2 \\
    &= \sigma^2  - \sigma^2 =0.
\end{align*}

\subsubsection{split环境(内嵌)}
\begin{equation*}
    \begin{split}
    (a + b)^4
      &= (a + b)^2 (a + b)^2      \\
      &= (a^2 + 2ab + b^2)
         (a^2 + 2ab + b^2)        \\
      &= a^4 + 4a^3b + 6a^2b^2 + 4ab^3 + b^4
    \end{split}
\end{equation*}

\subsubsection{带大括号的多行公式}
\paragraph{cases}
$$
    f=
    \begin{cases}
      x + y = z,  \\
      1 + 2 = 3.  \\
    \end{cases}
$$

\paragraph{array}
$$ F^{HLLC}=\left\{
\begin{array}{rcl}
F_L       &      & {0      <      S_L}\\
F^*_L     &      & {S_L \leq 0 < S_M}\\
F^*_R     &      & {S_M \leq 0 < S_R}\\
F_R       &      & {S_R \leq 0}
\end{array} \right. $$
    
\paragraph{aligned}
\begin{equation}
    \left\{
     \begin{aligned}
     \overset{.}x(t) &=A_{ci}x(t)+B_{1ci}w(t)+B_{2ci}u(t)  \\
     z(t) &=C_{ci}x(t)+D_{ci}u(t) \\
     \end{aligned}
     \right.
\end{equation}

\subsection{交换图}
\begin{sloppypar}
强烈推荐tikzcd-editor:\url{https://github.com/yishn/tikzcd-editor}
\end{sloppypar}

\begin{center}
\begin{tikzcd}
    T
    \arrow[drr, bend left, "x"]
    \arrow[ddr, bend right, "y"]
    \arrow[dr, dotted, "{(x,y)}" description] & & \\
    & X \times_Z Y \arrow[r, "p"] \arrow[d, "q"]
    & X \arrow[d, "f"] \\
    & Y \arrow[r, "g"]
    & Z
\end{tikzcd}
\end{center}

\begin{center}
    \begin{tikzcd}[row sep=scriptsize, column sep=scriptsize]
        & f^* E_V \arrow[dl] \arrow[rr] \arrow[dd] & & E_V \arrow[dl] \arrow[dd] \\
        f^* E \arrow[rr, crossing over] \arrow[dd] & & E \\
        & U \arrow[dl] \arrow[rr] & & V \arrow[dl] \\
        M \arrow[rr] & & N \arrow[from=uu, crossing over]\\
        \end{tikzcd}
\end{center}

\begin{center}
\begin{tikzpicture}[commutative diagrams/every diagram]
    \node (P0) at (90:2.3cm) {$X\otimes (Y\otimes (Z\otimes T))$};
    \node (P1) at (90+72:2cm) {$X\otimes ((Y\otimes Z)\otimes T))$} ;
    \node (P2) at (90+2*72:2cm) {\makebox[5ex][r]{$(X\otimes (Y\otimes Z))\otimes T$}};
    \node (P3) at (90+3*72:2cm) {\makebox[5ex][l]{$((X\otimes Y)\otimes Z)\otimes T$}};
    \node (P4) at (90+4*72:2cm) {$(X\otimes Y)\otimes (Z\otimes T)$};
    \path[commutative diagrams/.cd, every arrow, every label]
    (P0) edge node[swap] {$1\otimes\phi$} (P1)
    (P1) edge node[swap] {$\phi$} (P2)
    (P2) edge node {$\phi\otimes 1$} (P3)
    (P4) edge node {$\phi$} (P3)
    (P0) edge node {$\phi$} (P4);
\end{tikzpicture}
\end{center}
 %正文第三章
\section{表与图}这节用来展示表格与图片的插入。

\subsection{表格}
\par 本来LaTeX里表格的变化是非常多的,但鉴于学校要求用三线式,问题反而简单了。以下是一个例子:
\begin{table}[htbp]\center
    \caption{示例表格\\Example Table}
    \begin{tabular}{lcccccl}
     \toprule
     。。 & 。。 & 。。 & 。。 & 。。& 。。 & 。。\\
     \midrule
    。。 & 。。 & 。。 & 。。 & 。。& 。。 & 。。\\
    。。 & 。。 & 。。 & 。。 & 。。& 。。 & 。。\\
    。。 & 。。 & 。。 & 。。 & 。。& 。。 & 。。\\
    。。 & 。。 & 。。 & 。。 & 。。& 。。 & 。。\\
    。。 & 。。 & 。。 & 。。 & 。。& 。。 & 。。\\
     \bottomrule
    \end{tabular}
   \end{table}
如果你有使用更复杂的表格的需求,请自行查资料完成。

\subsection{插图}
由于这份模板不考虑多栏排版,所以格式要求中所述的半栏图大小要求我们不作演示。以下是一个通栏图的演示:
\begin{figure}[H]
    \centering
    \includegraphics[width=100mm]{example-image}
    \caption{图片测试(最小宽度)\\Image test (Minimal width)}
  \end{figure}

\begin{figure}[H]
    \centering
    \includegraphics[width=130mm]{example-image}
    %\includegraphics[width=130mm]{./figures/你自己的图像文件}
    \caption{图片测试(最大宽度)\\Image test (Maximal width)}
\end{figure}
\par 注意:这里为了减少图片上下的空白,使用了float宏包。 %正文第四章
\section{注释与引用}这节用来展示注释与引用。

\subsection{注释——脚注与尾注}
\subsubsection{脚注}
\par 这里是脚注测试\footnote{1111111111}这里是脚注测试这里是脚注测试这里是脚注测试\footnote{2222222222}这里是脚注测试这里是脚注测试这里是脚注测试这里是脚注测试这里是脚注测试这里是脚注测试这里是脚注测试这里是脚注测试这里是脚注测试这里是脚注测试这里是脚注测试这里是脚注测试这里是脚注测试这里是脚注测试这里是脚注测试\footnote{3333333333}这里是脚注测试这里是脚注测试这里是脚注测试这里是脚注测试这里是脚注测试这里是脚注测试这里是脚注测试这里是脚注测试这里是脚注测试这里是脚注测试这里是脚注测试这里是脚注测试

\textcolor{red}{\textbf{\uline{注意!正如这份演示中所出现的情况,若该页(也就是本文档中的前一页)剩余空间不大,不足以显示足够多的文档与脚注,那么该段文字就会被移至下一页而留下空白。目前我们尚未找到解决的方法,所以如果遇到了这个问题,请修改排版,以留下足够大的空间。}}}

\subsubsection{尾注}
\par 这里是尾注测试\endnote{伴随着互联网的发展以及新的网络应用的出现,互联网用户由单纯的“读”网页,向“读、写”网页,共同建设互联网发展,由此网上产生了大量带有用户主观感情的数据,从这些带...}这里是尾注测试这里是尾注测试这里是尾注测试这里是尾注测试\endnote{尾注测试2}这里是尾注测试这里是尾注测试这里是尾注测试这里是尾注测试这里是尾注测试这里是尾注测试这里是尾注测试这里是尾注测试这里是尾注测试这里是尾注测试这里是尾注测试这里是尾注测试这里是尾注测试这里是尾注测试这里是尾注测试\endnote{尾注测试3}这里是尾注测试这里是尾注测试这里是尾注测试这里是尾注测试这里是尾注测试这里是尾注测试这里是尾注测试这里是尾注测试这里是尾注测试

\par \textcolor{red}{\textbf{\uline{注意!endnotes宏包并不支持hyperref,也就是无法通过点击文中尾注标号以跳转到尾注。当然,这在打印出来的文档中并不会造成任何影响。}}}
\par \textcolor{blue}{\textbf{\uline{提示:尾注出现在全文最后。为了区分脚注与尾注的编号,我们在尾注编号前加上了“尾注”二字。}}}

\subsection{交叉引用}
\par 本模板使用cleveref宏包来进行交叉引用。使用的指令为$\backslash$cref$\{$label$\}$。例子如下:
\par 由\cref{thm2_1}我们可以知道XXXXXXXX。
\par 由\cref{lem2_2}我们可以知道XXXXXXXX。
\par 请注意,label是需要手工设置的,一般将label放在你需要引用的环境内即可(具体可见SectionB.tex)。

\subsection{文献引用的演示}
\par 本模板使用biblatex进行文献管理,这是一套相对较新的系统。另外,使用了hushidong制作的符合gb7714-2015标准的biblatex样式。在此对他的工作表示感谢,要完成这样的样式非常不容易。本模板中gb7714-2015.bbx与gb7714-2015.cbx即为他的作品,在这里打包发布以便使用。
\par 默认的bib文件位于~/reference/thesis-ref.bib,内容是由Wang Tianshu制作,在此仅作演示之用。关于bib文件的编写与管理请自行查找相关教程。
\par 下方的演示已经给出了正文中引用文献的基本方法,这与传统的cite命令是类似的。如有更多需求,请至\url{https://github.com/hushidong/biblatex-gb7714-2015}查找相关资料。
\par 文献\parencite{Yang_Hy200215}中提到xxxxxxx。
\par 文献\parencite{Joa1999}中提到yyyyyyy。
\par 文献\parencite{Altman1997}中提到zzzzzzz。
\par \textcolor{blue}{\textbf{\uline{本模板使用parencite而不是cite命令,因为这样能与脚注所产生编号进行区分。当然,如果你没有脚注或尾注,那么cite命令也是推荐使用的。}}}

 %正文第五章

\theendnotes %尾注(若没有尾注请将本行删除)
\ClearPageStyle

%生成参考文献
\phantomsection
\addcontentsline{toc}{section}{参考文献}
\printbibliography[title={\centerline{\bfseries\sffamily \zihao {-3}参考文献}}]
\ClearPageStyle

%生成附录
\phantomsection
\addtocontents{toc}{\setcounter{tocdepth}{1}}
\addcontentsline{toc}{section}{附录}
\setcounter{subsection}{0}
\ctexset { subsection = { name={,},number={\arabic{subsection}},format={\rmfamily \zihao {5}} } }
\ctexset { subparagraph = { name={(,)},number={\arabic{subparagraph}},format={\rmfamily \zihao {5}},indent=2em } }
\section*{\centerline{\bfseries \sffamily \zihao{-3} 附录}}

\subsection{实验数据}
\subparagraph{吐槽}
2019年的样板做得实在太烂了

\subsection{调查结果}
23333333333333333333333333333333333333333
\ClearPageStyle

\makeacknowledgement %生成感谢

\end{document} 
