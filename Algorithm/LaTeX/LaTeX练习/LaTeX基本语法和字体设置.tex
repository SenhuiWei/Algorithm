%!TEX program = xelatex 
\documentclass[10pt]{article} % [10pt]设置normal size大小,且只有10 11 12pt 
\usepackage{xunicode, xltxtra, ctex, lmodern,CJKfntef} % ctex是我后来加的 
\usepackage{fontspec}
\setmonofont{Consolas}


% 利用newcommand定义新命令 相当于定义函数或typedef
\newcommand{\myfont} {\textbf{\textsf{Fancy Text}}} 
\newcommand{\degree} {^\circ}  % \circ表示°,\angle表示∠

\title{\rmfamily My First Document} 
\author{Wilson79} 
\date{\today}

\begin{document} 
\section{强调}
	% 几个常见的强调命令有\uline,\uuline,\uwave,\sout,\dotuline
	\begin{flushleft}

	\uline{Fashion}\\
	\uuline{Fashion}\\
	\uwave{Fashion}\\
	\sout{Fashion}\\
	\dotuline{Fashion}\\

	\end{flushleft}


	{\myfont $ 12\degree$}

	% 字体族设置(罗马字体、无衬线字体、打字机字体)
	% 法一 字体设置命令
	\textrm{Roman Family} 
	\textsf{Sans Serif Family}
	\texttt{Typewriter Family}

	% 法二 字体设置声明 作用于后续所有字体
	\rmfamily Roman Family
	\sffamily Sans Serif Family
	{\ttfamily Typewriter Family}

	% 字体系列设置(粗细、宽度)
	\textbf{Boldface Series}
	\textmd{Medium Series}

	{\bfseries Boldface Series}
	{\mdseries Medium Series}

	
	
	% 字体形状(直立、斜体、伪斜体、小型大写)
	\textup{Upright Shape} \textit{Italic Shape}
	\textsl{Slanted Shape} \textsc{Small Caps Shape}

	{\upshape Upright Shape} {\itshape Italic Shape} 
	{\slshape Slanted Shape} {\scshape Small Caps Shape}

	% 中文字体 需使用ctex宏包
	{\songti 宋体} {\heiti 黑体} \quad {\fangsong 仿宋}
	{\kaishu 楷书}

	% 对于中文默认粗体用黑体表示,斜体用楷书表示,斜体只对英文有效
	中文字体的\textbf{粗体 }与\textit{Italic 斜体} 


	% 字体大小
	{\tiny Hello New Life 你好新生活} \\
	{\scriptsize Hello New Life 你好新生活} \\
	{\footnotesize Hello New Life 你好新生活} \\
	{\small Hello New Life 你好新生活} \\
	{\normalsize Hello New Life 你好新生活} \\
	{\large Hello New Life 你好新生活} \\
	{\Large Hello New Life 你好新生活} \\
	{\LARGE Hello New Life 你好新生活} \\
	{\huge Hello New Life 你好新生活} \\
	{\Huge Hello New Life 你好新生活} \\

	% 中文字号设置命令 ctex
	{\zihao{5} 你好!}
	{\zihao{-0} 你好!}



	%%%%%%%%%%%%%%%%%%%%%%%%%%%%%%%%%%%%%%%%%%%%
	\maketitle{第一章} 

	\section{Introduction} 

	{\ttfamily \itshape This is where you will write your content. }

	% 这里是段间注释	

	% 不带标号的公式
	\rmfamily Let $f(x)$ be defined by the formula $$f(x)=3x^2+x-1$$ which is a polynomial of degree 2.

	% 带标号的公式 不能再写$$
	\begin{equation}
	a^2 = b^2+c^2
	\end{equation}

\end{document}


% 注释区:
% LaTeX的思想是内容与格式的分离,因此不建议在文档中使用大量命令,而是像函数一样定义新命令 \newcommand

% 1. letter里面没有\maketitle类
% 2. latex不需要的代码不建议删除,而是注释掉,以便以后有用
% 3. 单$模式表示行内公式,双$$模式表示行间公式,可以另起一行居中显示数学公式


% 字体属性
% 在LaTeX中,一个字体有5种属性

% 字体族
% 当遇到另一个字体声明是会用新的字体声明,{}可以分组,限定生效范围
% \quad表示空格
% \\ 表示换行但不首行缩进  所以它只是换行,不会产生新段落
% 空行 表示换行+首行缩进. 
% \par 等效于空行


% 终端输入
% texdoc ctex 
% texdoc lshort-zh
% 可以查看帮助文件

% 每次写latex先写好提高