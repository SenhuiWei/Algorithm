\documentclass{article} 
\usepackage{xunicode, xltxtra, lmodern, ctex} 
\usepackage{amsthm, amssymb, amsmath}
\usepackage[a4paper,left=3cm,right=3cm,top=2cm,bottom=2cm]{geometry} 

\title{Math Type} 
\author{Wilson79} 
\date{\today}

\begin{document} 
    \tableofcontents % 产生文档目录
    \maketitle %{第一章} 
    
    $f(x)=\frac{1}{\sqrt{2 \pi \sigma x}} e^{-\frac{(x-\mu)^{2}}{2 \sigma^{2}}}$

\section{数学符号汇总!}
    
    % 通过\limits把下标调到正侧
    $\sum _{i=1}^{n}$ % 行内公式的那些下标一般是偏右侧的,而行间公式会在正侧
    $\sum \limits_{i=1}^{n}$
    $\lim\limits _{x \rightarrow 0+} x^2 + x$ 

    \[
    \begin{aligned} 
        \geq \\ % 大于等于 great
        \leq \\ % 小于等于 less
        \neq \\ % 不等于   not
        \sim \\ % ~
        \varepsilon \\
        \forall \\ % 任意
        \exists \\ % 存在
        \Rightarrow \\ 
        \vert x \vert \\ % |x|
        \Vert x \Vert \\ % ||x||
        \sin x \\ % sinx它们中间需要加个空格
        \{  \\  % { 需要转义
        d \quad \mathrm{d} \\ %斜体与非斜体
        \rightarrow \\ % 右箭头
        \quad %\quad表示拉开一段距离
        \sum\limits_{i=1}^{n} \\ % 加个limits表示下标位置在正上方
        =& o+p+q+r+s \\ 
        =& t+u+v+x+z 
    \end{aligned}
    \]
    

    

\section{行内公式} %小节
    我们来看公式$a+b=2$ % 行内公式 $$ 或 \( \) 或 \begin{math} \end{math}

\section{行间公式}
    \[a^2 = b ^2 + c^2 \] % 行间公式
    $$a^2 = b ^2 + c^2$$
    \begin{displaymath}
    a^2 = b ^2 + c^2
    \end{displaymath}

\section{自动编号公式equation}
    \begin{equation} % 需要amsmath宏包
     c^2 = b ^2 + d^2 
    \end{equation}

    % 交叉引用
    详见公式\ref{eq:a}
    \begin{equation} 
    c^2 = b ^2 + d^2 \label{eq:a}
    \end{equation}

    \begin{equation} 
     c^2 = b ^2 + d^2 \tag{*} 
    \end{equation}

\section{不自动编号公式equation*}  
    \begin{equation}
    d^2 = a ^2 + c^2 \notag
    \end{equation}
    
    \begin{equation*}
    d^2 = a ^2 + c^2 
    \end{equation*}

\section{定理环境}
    % 宏包amsthm
    \begin{proof}
    For simplicity, we use
    \[
    E=mc^2
    \]
    That's it.
    \end{proof}

\section{上标下标}
    $$3x^{x_{20} + 3} - x + 2 = 0$$
    $$\beta_0,a_1,...,a_{100}$$

\section{希腊字母} 
    $\alpha$
    $\pi$
    $\beta$
    $\gamma$

    $\beta^2 = 16$

\section{数学函数}
    $\log$
    $\sin$
    $\arccos x$
    $\ln x$

    $\sin^2 x+\cos^2 x = 1$
    $\log_2 x$

    $\sqrt{x ^ 2 + y ^ 2}$
    $\sqrt{2 + \sqrt[3]{9}}$ % 可选参数决定开方次数

\section{分式}
    大约是原体积的$3/4$
    大约是原体积的$\frac{3}{4}$

    $\frac{\sqrt{x-1}}{\sqrt{x+1}}$

    $\sqrt{\frac{x}{x^{11} - x + 3}}$



\section{多行公式}
    % 目前最常用的是 align 环境,它将公式用 & 隔为两部分并对齐。分隔符通常放在等号左边
    % align默认会加上编号,用notag去掉
    \begin{align} 
    a & = b + c \notag \\ 
    &=d+e 
    \end{align}
    
    % aligned可以不加编号,它需要与\[ \]嵌套使用
    \[
    \begin{aligned}  % 不能有多余的空行,会报错
    a+b+c+d+& e+f+g+h+i \\
    =& j+k+l+m+n \\
    =& o+p+q+r+s \\ 
    =& t+u+v+x+z \end{aligned}
    \]

\section{矩阵}
    % array环境(类似于表格环境tabular)
    \[
    a + b \\ + c\quad % \\行间公式不能换行
    \begin{array}{l | l}
    {1} & {2} \\ 
    \hline
    {3} & {4}
    \end{array} 
    \left[\begin{array}{cccc}
    {x_{11}} & {x_{12}} & {\dots} & {x_{1 n}} \\ 
    {x_{21}} & {x_{22}} & {\dots} & {x_{2 n}} \\ 
    {\vdots} & {\vdots} & {\ddots} & {\vdots} \\ 
    {x_{n 1}} & {x_{n 2}} & {\dots} & {x_{n n}}
    \end{array}\right]
    \]

    % 带下标的矩阵
    \[
    A = \begin{bmatrix}
        a_{11} & \dots & \text{\Large5}\\ % \text在数学模式下临时切换为文本模式
        & \ddots  & \vdots \\ % 没有反斜省略号
        1 &  & 2    
    \end{bmatrix}_{n \times n} % \times表示×
    \]

    % 行内小矩阵(smallmatrix)
    复数$z = (x, y)$也可用矩阵 
    \begin{math}
    \left(\begin{smallmatrix}
    x & -y \\
    y & x   
    \end{smallmatrix}\right) % 定界符加在)左边
    \end{math}

    % 代码    
    \begin{verbatim}
    class Solution {
    public:
        int numberOfSubarrays(vector<int>& nums, int k) {
            // use the prefix sum
            unordered_map <int, int> hash;
             
            int ans = 0, tot = 0;
            hash[0] = 1;
            for (auto x : nums) {
                if (x & 1) x = 1;
                else x = 0;
                tot += x;
                // add the number of prefixes that add up to tot - k
                ans += hash[tot - k]; 
                hash[tot] ++;
            }
            
            return ans;
        }
    };
            
    \end{verbatim}


\section{复杂公式例子}
    \[
    (25)\quad y=\left(x-a_{1}\right)^{a_{1}}\left(x-a_{2}\right)^{a_{2}} \cdots\left(x-a_{n}\right)^{a_{n}}
    \]

    \[
    \lim _{x \rightarrow \infty} \frac{x^{2}-5}{x^{2}-1}=1
    \]

    $$\lim _{x \ddots \infty} {f(x)+ 2} = 1$$

    % \left \right 定界符 保证你的括号大小也会随公式去变化
    \[1 + \left(\frac{1}{1-x^{2}}
    \right)^3 \qquad
    \left.\frac{\partial f}{\partial t}
    \right|_{t=0}\]

    % 大于号小于号这些你通过mathpix去反着学,很快的
    \[(1 + x + x^2)^{-1} \leqslant (1 + x + x^2)^{\sin\frac{1}{x}} \geqslant (1+x+x^2)^{1}\] % leqslant需要宏包amssymb

    \[
    \begin{aligned} y &=\ln \frac{(\sqrt{1+x}-\sqrt{1-x})^{2}}{2 x}=\ln \frac{1-\sqrt{1-x^{2}}}{x} \\ &=\ln (1-\sqrt{1-x^{2}})-\ln x \end{aligned}
    \]


    \[
    \begin{array}{l}{(26) y^{\prime}=\frac{1}{\sqrt{a^{2}-b^{2}}} \frac{1}{\sqrt{1-\left(\frac{a \sin x+b}{a+b \sin x}\right)^{2}}}} \\ {\times \frac{a \cos x(a+b \sin x)-b \cos x(a \sin x+b)}{(a+b \sin x)^{2}}} \\ {=\frac{\sqrt{a^{2}-b^{2}} \cos x}{|a+b \sin x| \sqrt{a^{2}-b^{2}}|\cos x|}=\frac{\cos x}{|a+b \sin x||\cos x|}}\end{array}
    \]

    \[
    H(Y | X)=\sum_{x \in \mathcal{X}, y \in \mathcal{Y}} p(x, y) \log \left(\frac{p(x)}{p(x, y)}\right)
    \]

    \[
    \Gamma_{\epsilon}(x)=\left[1-e^{-2 \pi \epsilon}\right]^{1-x} \prod_{n=0}^{\infty} \frac{1-\exp (-2 \pi \epsilon(n+1))}{1-\exp (-2 \pi \epsilon(x+n))}
    \]

    ${a,b,c}\neq \{a,b,c\}$ % \neq 不等于

\section{积分}
    \[
    \lim _{n \rightarrow \infty} \int_{E} f_{n}(x) \mathrm{d} x=0
    \]

    $x^{2} \geq 0 \qquad
    \text{for \textbf{all} }
    x\in\mathbb{R}$

    \[
    \sup _{\varphi \leq f}\left\{\int_{E} \varphi \mathrm{d} x\right\}
    \]

    % 较复杂的下标
    \[
    \int_{E} f(x) \chi_{\{x \in E: f(x)>N\}}(x) \mathrm{d} x<\varepsilon
    \]

    % 实变函数的公式
    \[
    \sum_{n \geq 0} \int_{E_{n}}|f(x)| \mathrm{d} x=\int_{\cup_{n \geq 0} E_{n}}|f(x)| \mathrm{d} x=\int_{\mathbb{R}}|f(x)| \mathrm{d} x<\infty
    \]
    
    % 集合
    \[
    \{x \in[a, b]: f(x) \neq 0\}=\{x \in[a, b]: f(x)>0\} \cup\{x \in[a, b]: f(x)<0\}
    \]
\end{document}